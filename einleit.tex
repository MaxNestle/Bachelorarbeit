

\chapter{Einleitung}
\label{cha:einleitung}

\section {Aufgabenstellung und Zielsetzung}

Ziel der Bachelorarbeit ist die Erstellung einer Messenger Anwendung, bei dem zwei Personen Daten empfangen und verschicken können. Dabei soll die Kommunikation über ein Netzwerk statt finden und steganografisch verschlüsselt werden.\\ Die Daten sollen nicht mit einem mathematischen Verfahren verschlüsselt werden, sondern in ein oder mehreren Protokollen ,,versteckt'' eingebettet und übertragen werden. Dazu soll ein optimales Verfahren zur Dateninfiltration und -exfiltration gefunden werden. Als Verfahren können hier zum Beispiel Covert- Channels eingesetzt werden. Das Verfahren sollte unauffällig, für Dritte schwer zu interpretieren und mit größt möglicher Übertragungsrate senden. Optimal ist ein ähnliche Sicherheit zu gewährleisten wie mit einer mathematischen Verschlüsselung.\\
Ziel ist es außerdem jedes Dateiformat übertragen zu können. Das resultierende Programm soll in der Lage gleichzeitig Server und Client zu sein, dass bedeutet es kann mit dem gleichen Programm gesendet und empfangen werden.\\
Eine einfache GUI soll dem Benutzter das senden und empfangen der Daten so einfach wie möglich machen.\\
Wünschenswert wäre zudem noch ein möglichst einfacher Verbindungsaufbau der ohne den direkten Austausch der IP- Adressen statt finden kann.


%%% Local Variables: 
%%% mode: latex
%%% TeX-master: "Bachelorarbeit"
%%% End: 
