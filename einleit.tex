

\chapter{Einleitung}
\label{cha:einleitung}


\section {Motivation}

In der heutigen Zeit wird die Datensicherheit immer wichtiger, da immer mehr personenbezogene Daten im Internet preisgegeben werden. Um die Daten\"ubertragung zu sichern wird meistens ein asymetrisches Verschl\"usselungsverfahren verwendet.\\
Dieses Verfahren bieten zwar ein hohes Ma{\ss} an Sicherheit, hat aber auch Nachteile, wie zum Beispiel eine Erh\"ohung der Rechenzeit, die Verwaltung eines Key-Managers und die Bedrohung durch einen ,,Man-in-the-Middle'' Angriff.\\ 
Eine M\"ogliche Alternative w\"are beispielsweise die Steganographie. Dies ist die Kunst, Daten in legitimierten Datenkan\"alen zu verstecken ohne einen Verschl\"usselung anzuwenden.\\
Steganographie ist zudem sehr unauff\"allig, da in unverschl\"usselten Daten keine sensiblen Daten vermutet werden. \\
So w�re beispielsweise eine Anwendung denkbar bei der ein Polizeipr�sidium mit ihren verdeckten Ermittlern kommunizieren will. Da die Polizei davon ausgehen muss, dass die Kommunikation abgeh�rt wird, w�rde einen Verschl�sselte Kommunikation zu viel Aufsehen erregen. Zudem kann es sein, dass die Verschl�sselung auch schon geknackt wurde. \\
Hier kommt dann die Steganografie ins Spiel, welche es m�glich macht eine legitimen und unauff�lligen Kanal f�r die Daten�bertagung zu verwenden. So ein Kanal k�nnte der Stream beim Schauen eines Videos oder die Nachrichten einer Webseite sein.\\
Die Steganographie bietet gerade deswegen, da sie oft hinter den gro�en Verschl�sselungsverfahren, in Vergessenheit ger�t eine sehr gut Methode um hoch sensible Daten zu versenden.

\section {Aufgabenstellung und Zielsetzung}

Ziel der Bachelorarbeit ist es, bei dem in der Motivation bereit beschriebenen Szenario, der Polizei eine Kommunikationsm�glichkeit zu schaffen. Dabei soll es m�glich sein, dass Anweisungen, Treffpunkte aber auch Bilder an den verdeckten Ermittler �bertragen werden k�nnen.
Dabei soll die Kommunikation \"uber ein Netzwerk stattfinden und steganografisch verschl\"usselt werden.\\ Die entstehende Anwendung soll als ein ,,Proof of Concept'' dienen und die M\"oglichkeiten der Steganografie veranschaulichen.\\ Die Daten sollen nicht mit einem mathematischen Verfahren verschl\"usselt werden, sondern in ein oder mehreren Protokollen ,,versteckt'' eingebettet und \"ubertragen werden. Dazu soll ein optimales Verfahren zur Dateninfiltration und -exfiltration gefunden werden. Das Verfahren sollte unauff\"allig, f\"ur Dritte schwer zu interpretieren und mit gr\"o{\ss}t m\"oglicher \"Ubertragungsrate senden. Optimal w\"are ein \"ahnliche Sicherheit zu gew\"ahrleisten, wie mit einer mathematischen Verschl\"usselung.\\
Ziel ist es au{\ss}erdem jedes Dateiformat \"ubertragen zu k\"onnen. \\


\section{Aufbau}
Als erstes folgt die Einf�hrung in die Grundlagen.
Danach besch�ftigt sich die Arbeit mit der Findung nach einem optimalen steganographischen Verfahren, mit dem die Daten �bertragen werden sollen.
Anschlie�end wird dieses Verfahren in einer realen Anwendung �bertragen.
Als letztes wird die Anwendung bewertet und ein Fazit gezogen.

\section{Eigene Leistung}
Es werden steganografische Verfahren bewertet und das Optimum ausfindig gemacht. Aus dem gefundenen Ergebnis wird einen Anwendung als ,,Proof of Concept'' implementiert die passend zur Zielsetzung die Datenkommunikation �bernimmt.
Das Programm wird danach evaluiert und auf m�gliche Anwendungsgebiete getestet.


%%% Local Variables: 
%%% mode: latex
%%% TeX-master: "Bachelorarbeit"
%%% End: 
