

\chapter{Einleitung}
\label{cha:einleitung}


\section {Motivation}

In der heutigen Zeit wird die Datensicherheit immer wichtiger, da immer mehr personenbezogene Daten im Internet preisgegeben werden. Um die Daten\"ubertragung zu sichern wird meistens ein asymetrisches Verschl\"usselungsverfahren verwendet.\\
Dieses Verfahren bieten zwar ein hohes Ma{\ss} an Sicherheit, hat aber auch Nachteile, wie zum Beispiel eine Erh\"ohung der Rechenzeit, die Verwaltung eines Key-Managers und die Bedrohung durch einen ,,Man-in-the-Middle'' Angriff.\\ 
Eine M\"ogliche Alternative w\"are beispielsweise die Steganographie. Dies ist die Kunst, Daten in legitimierten Datenkan\"alen zu verstecken ohne einen Verschl\"usselung anzuwenden.\\
Steganographie ist zudem sehr unauff\"allig, da in unverschl\"usselten Daten meistens keine sensiblen Daten vermutet werden.

\section {Aufgabenstellung und Zielsetzung}

Ziel der Bachelorarbeit ist die Erstellung eines Messengers, bei dem zwei Personen Daten empfangen und verschicken k\"onnen. Dabei soll die Kommunikation \"uber ein Netzwerk stattfinden und steganografisch verschl\"usselt werden.\\ Die Daten sollen nicht mit einem mathematischen Verfahren verschl\"usselt werden, sondern in ein oder mehreren Protokollen ,,versteckt'' eingebettet und \"ubertragen werden. Dazu soll ein optimales Verfahren zur Dateninfiltration und -exfiltration gefunden werden. Als Verfahren k\"onnen hier zum Beispiel Covert- Channels eingesetzt werden. Das Verfahren sollte unauff\"allig, f\"ur Dritte schwer zu interpretieren und mit gr\"o{\ss}t m\"oglicher \"Ubertragungsrate senden. Optimal w\"are ein \"ahnliche Sicherheit zu gew\"ahrleisten, wie mit einer mathematischen Verschl\"usselung.\\
Ziel ist es au{\ss}erdem jedes Dateiformat \"ubertragen zu k\"onnen. Das resultierende Programm soll in der Lage sein, gleichzeitig Server und Client zu sein, was bedeutet, dass mit dem gleichen Programm gesendet und empfangen werden kann.\\
Eine einfache GUI soll dem Benutzter das Senden und Empfangen der Daten so einfach wie m\"oglich machen.\\
W\"unschenswert w\"are ein m\"oglichst einfacher Verbindungsaufbau, der ohne den direkten Austausch der IP- Adressen statt findet.\\


%%% Local Variables: 
%%% mode: latex
%%% TeX-master: "Bachelorarbeit"
%%% End: 
