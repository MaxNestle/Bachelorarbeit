
%% Vorlage fuer Studien- und Bachelorarbeiten an der HS Ravensburg-Weingarten
%% Benoetigt wird KOMA-Skript ab Version 2.8j vom 30.07.2001
%% Vor der Veraenderung irgendwelcher Einstellungen wird dringend empfohlen
%% die Anleitung zum KOMA-Skript (scrguide) zu konsultieren !!
%%
%% Bilder bitte nicht mit Endung einbinden, so ist die Erzeugung von
%% DVI, PS und PDF problemlos m�glich!
%% J. Moehler, 2008-12-10 
%%%%%%

%% Dokumentendefinition
\documentclass[
   12pt,                % Schriftgroesse 12pt
   a4paper,             % Layout fuer Din A4
   german,              % deutsche Sprache, global   
%  twoside,             % Layout fuer beidseitigen Druck
	 oneside,						  % Layout f�r einseitigen Druck
hidelinks,
   headinclude,         % Kopfzeile wird Seiten-Layouts mit beruecksichtigt
   headsepline,         % horizontale Linie unter Kolumnentitel
   plainheadsepline,    % horizontale Linie auch beim plain-Style
   BCOR12mm,            % Korrektur fuer die Bindung
   DIV18,               % DIV-Wert fuer die Erstellung des Satzspiegels, siehe scrguide
   halfparskip,         % Absatzabstand statt Absatzeinzug
   openany,             % Kapitel k�nnen auf geraden und ungeraden Seiten beginnen
   bibtotoc,            % Literaturverz. wird ins Inhaltsverzeichnis eingetragen
   pointlessnumbers,    % Kapitelnummern immer ohne Punkt
   tablecaptionabove,   % korrekte Abstaende bei TabellenUEBERschriften
   fleqn,               % fleqn: Glgen links (statt mittig)
%   draft               % Keine Bilder in der Anzeige, overfull hboxes werden angezeigt !Auskommentieren f�r Test-Compile!
]{scrbook}[2001/07/30]  % scrbook-Version mind. 2.8j von 2001/07/30

%% Pakete f�r n�tzliche Dinge


\usepackage{color} 							 % Schriftfarben verwenden
\usepackage{ngerman}             % neue deutsche Rechtschreibung
\usepackage[ngerman]{translator} % �bersetzung 
\usepackage[ansinew]{inputenc}   % Input-Encodung: ansinew fuer Windows
%\usepackage[latin1]{inputenc}   % Input-Encodung: latin1 fuer Unix
\usepackage[T1]{fontenc}         % T1-kodierte Schriften, korrekte Trennmuster fuer Worte mit Umlauten
\usepackage{ae}                  % F�r PDF-Erstellung
\usepackage[hang]{caption2}      % mehrzeilige Captions ausrichten


\usepackage[centertags]{amsmath} % AMS-Mathematik, centertags zentriert Nummer bei split
\usepackage{latexsym}            % verschiedene Symbole
\usepackage{textcomp}            % verschiedene Symbole
 \usepackage{microtype}          % bessere Optik     
\usepackage{graphicx}            % zum Einbinden von Grafiken
\usepackage{float}               % u.a. genaue Plazierung von Gleitobjekten mit H
% \usepackage{pstricks}          % PostScript Macros
% \usepackage{lscape}            % Seite im Querformat bei Erhalt der Kopfzeile
% \usepackage{verbatim}          % Quellcode einbinden (\verbatiminput)
% \usepackage{multicol}          % Mehrspaltiger Text 

%%% Literatur und sonstige Referenzen
\usepackage{cite}              % Sortierte und zusammengefasste Zitatnummern 
\usepackage{url}							 % URL in Literatur wird unterst�tzt
\usepackage{varioref}          % Verbesserte Referenzen
\usepackage{hyperref}          % Verlinkte Verzeichnisse
% \usepackage[numbers, sort]{natbib} % DIN Literaturverzeichnis; nicht zusammen mit cite verwenden!

%% Index
\usepackage{makeidx}						 % Index verwenden
\makeindex											 % index erstellen

%% Zeilenabstand
\usepackage{setspace}            % Zeilenabstand einstellbar
\onehalfspacing                  % eineinhalbzeilig einstellen

%% Kopf- und Fusszeilen
\usepackage{scrpage2}                             % Kopf und Fusszeilen-Layout 
\renewcommand{\headfont}{\normalfont\sffamily}    % Kolumnentitel serifenlos
\renewcommand{\pnumfont}{\normalfont\sffamily}    % Seitennummern serifenlos
\pagestyle{scrheadings}
\ihead[]{\headmark}               % Kolumnentitel immer oben innen
\chead[]{}                        % oben Mitte
\ohead[\pagemark]{\pagemark}      % Seitennummern immer oben aussen
\ofoot[]{}                        % Fusszeile aussen
\cfoot[]{}                        % Fusszeile Mitte
\ifoot[]{}                        % Fusszeile innen

%% Fussnotenz�hler
\usepackage{chngcntr}              % Paket um Counter zu steuern

\usepackage{threeparttable}        % Tabelle mit Fussnoten

%% Namen von Verzeichnissen definieren
\renewcommand{\bibname}{Literatur}               % Literaturverzeichnis wird zu Literatur
\renewcommand{\figurename}{Bild}                 % Abbildung wird zu Bild
\renewcommand{\listfigurename}{Bildverzeichnis}
\renewcommand{\indexname}{Stichwortverzeichnis}

%% Schrift mit Serifen auch fuer Ueberschriften benutzen
%\renewcommand*{\sectfont}{\bfseries}
%\renewcommand*{\descfont}{\bfseries}

\typearea[current]{current}        % Neuberechnung des Satzspiegels mit alten Werten nach �nderung von Zeilenabstand,etc

% -- Glossar --
\usepackage[toc, acronym]{glossaries} % Glossareintr�ge, muss nach hyperref (insofern dies verwendet wird) geladen werden
% (aufgrund der Seitenzahlverlinkung im Glossarverzeichnis), (ben�tigt die Packete
% "xkeyval" und "supertabular", welche dann automatisch eingebunden werden)
\newglossary[slg]{symbolslist}{syi}{syg}{Symbolverzeichnis} %Ein eigenes Symbolverzeichnis erstellen
\renewcommand*{\glspostdescription}{} %Den Punkt am Ende jeder Beschreibung deaktivieren
\makeglossaries % erstellt ein Glossar (Verzeichnis f�r Begriffserkl�rungen,
% z.B.: Abk�rzungen)
% Glossareintr�ge (MUSS f�r JEDEN Glossareintrag �berarbeitet werden):
%% Symbole:
%%\newglossaryentry{symb:Name}{name=Symbolname, description={Beschreibung}, sort=alphabetisches Wort f�r die Einreihung, type=symbolslist}
\newglossaryentry{symb:Pi}{
name=$\pi$,
description={Die Kreiszahl.},
sort=symbolpi, type=symbolslist
}
%%Abk�rzungen:
%%\newacronym{Referenz}{Abk�rzung}{Beschreibung}
\newacronym{BSP}{BSP}{Beispiel}


%%Eine Abk�rzung mit Glossareintrag:
%%\newacronym{Referenz}{Abk�rzung}{Beschreibung\protect\glsadd{glos:Referenz}}
\newacronym{AD}{AD}{Active Directory\protect\glsadd{glos:AD}}

%%Glossareintrag:
%%\newglossaryentry{glos:Referenz}{name=Name, description={Beschreibung}}
\newglossaryentry{glos:AD}{
name=Active Directory,
description={Active Directory ist in einem Windows Server 2000, Windows
Server 2003, oder Windows Server 2008-Netzwerk der Verzeichnisdienst, 
der die zentrale Organisation und Verwaltung aller Netzwerkressourcen erlaubt. Es
erm�glicht den Benutzern �ber eine einzige zentrale Anmeldung den
Zugriff auf alle Ressourcen und den Administratoren die zentral
organisierte Verwaltung, transparent von der Netzwerktopologie und
den eingesetzten Netzwerkprotokollen. Das daf�r ben�tigte
Betriebssystem ist entweder Windows Server 2000, 
Windows Server 2003, oder Windows Server 2008, welches auf dem zentralen
Dom�nencontroller installiert wird. Dieser h�lt alle Daten des
Active Directory vor, wie z.B. Benutzernamen und
Kennw�rter.}
}

\newglossaryentry{glos:Glossareintrag}{name=Glossareintrag, description={Erweiterte Informationen zum
einem Wort oder einer Abk�rzung, �hnlich einem Eintrag im Duden.}}




%%% Local Variables: 
%%% mode: latex
%%% TeX-master: "Bachelorarbeit"
%%% End: 
\usepackage[left = 2.54cm,right = 2.54cm]{geometry}

%%% PDF-Erzeugung: pdflatex statt latex aufrufen!
%% BEI WINDOWS UND TEXNICCENTER AUSKOMMENTIERT LASSEN !!! 
%\pdfoutput=1                  % PDF-Ausgabe
%\usepackage[pdftex, a4paper,  % muss letztes Package sein!
%     pdftitle={Titel der Arbeit},%
%     pdfauthor={Name Autor},%
%     pdfsubject={Studien- bzw. Diplomarbeit},%
%     pdfkeywords={Stichwort zur Arbeit},%
%    ]{hyperref} % 



%\graphicspath{{figs/}{bilder/}}    % Falls texinput nicht gesetzt -> Bildverzeichnisse

%\includeonly{}


%%%%%%%%%%%%%%%%%%%%%%%%%%%%%%%%%%%%%%%%%%%%%%%%%%%%%%%%%%%%%%%
\begin{document}

\pagenumbering{Roman}           % Nummerierung R�misch Start bei I

%% Deckblatt fuer Studien- und Diplomarbeiten am der
%% Hochschule Weingarten

\thispagestyle{empty}
%~
{
\normalsize\fontfamily{phv} \fontsize{12pt}{10}\selectfont 
\vspace{-1cm}
\begin{minipage}[b]{9.4cm}
{\fontsize{13pt}{13} \selectfont%
Hochschule\\[1ex]
Ravensburg-Weingarten}\\[1ex]
\end{minipage}
}
\begin{minipage}[b]{10cm}
\includegraphics*[height=2.7cm]{bilder/HSLogoWGd}
\end{minipage}


\vspace{10mm}
 
\hrule 
\vspace{1cm}
{
\fontseries{b} \fontsize{20pt}{20}  \selectfont%
\begin{center}
{Messenger �ber einen verborgenen Netzwerkkanal} % Titel der Arbeit
\end{center}
}

\begin{center}
\large \textbf{Bachelorarbeit} % Hier Praxisarbeit, Studienarbeit, Bachelorarbeit, Dokumentation zu Seminar, etc. eintragen
\end{center}

\begin{center}
\textbf{Wintersemester 2018/19} % Hier den Zusatz wie Fach oder Semester eintragen
\end{center}

\vspace{5mm}

\begin{center}
im Studiengang Angewandte Informatik % Hier den Studiengang eintragen
\end{center}

\begin{center}
an der Hochschule Ravensburg - Weingarten 
\end{center}
\begin{center}

\end{center}
\vspace{5mm}
\begin{center}
von
\end{center}




\begin{center}
{\fontsize{12pt}{12} \selectfont%
\begin{tabular}{ll}
Maximilian Nestle & {Matr.-Nr.: 27427}\\[0.5ex] % Hier den Autor und die Matrikelnummer statt xxxxx eintragen
%Autor 2 & \textcolor{red}{Matr.-Nr.: xxxxx}\\[0.5ex] % F�r weitere Autoren Zeilen auskommentieren bzw. kopieren und ausf�llen
%Autor 3 & \textcolor{red}{Matr.-Nr.: xxxxx}\\[0.5ex]
Abgabedatum :& \today   % Das Abgabedatum wird gleichgesetzt mit dem Datum der letzten Compilierung. Statt \today kann auch Datum von Hand geschrieben werden
\end{tabular}
}
\end{center}
                               

\vspace{1cm}

\vspace{1cm}
\hrule


%%% Local Variables: 
%%% mode: latex
%%% TeX-master: "Bachelorarbeit"
%%% End: 

                % Deckblatt Nummerierung unterdr�ckt (In Deckblatt festgelegt)
%%\cleardoubleemptypage         % Die Eidesstattliche Erklaerung auf einer rechten Seite beginnen
%% Eidestattliche Erkl�rung %%
%% Die Erkl�rung sollte nach dem Deckblatt fest abgeheftet werden
\addchap*{Erkl�rung} %* nicht entfernen, sonst erh�lt Erkil�rung eine Nummer und erscheint im Inhaltsverzeichnis

\thispagestyle{empty} %Keine Seitenzahl, keine Kopf- und Fusszeile

Hiermit erkl"are ich, dass ich die vorliegende Arbeit mit dem Titel   % ein Autor
% Hiermit erkl"aren wir, dass wir die vorliegende Arbeit mit dem Titel \newline % mehrere Autoren
\begin{center}
\textbf{
Konstruktion und Implementierung eines versteckten Datenkanals mit Hilfe der Steganographie oder Covert-Channels}
\end{center}
selbst"andig angefertigt, nicht anderweitig zu Pr�fungszwecken vorgelegt, keine anderen als die angegebenen Hilfsmittel benutzt und w�rtliche sowie sinngem��e Zitate als solche gekennzeichnet habe.\newline  % ein Autor
%selbst"andig angefertigt, nicht anderweitig zu Pr�fungszwecken vorgelegt, keine anderen als die angegebenen Hilfsmittel benutzt und w�rtliche sowie sinngem��e Zitate als solche gekennzeichnet haben.\newline  % mehrere Autoren

\begin{flushleft}
Weingarten, \today % Ort eintragen, /today kann durch Datum 2009-10-21 oder 21.10.2009 ersetzt werden
\end{flushleft}

%%% Unterschriftenblock f�r einen Autor
\begin{tabular}{l}   
Maximilian Nestle        \\% Hier Autor eintragen
 \\
------------------------------------ \\
\end{tabular}

%%% Unterschriftenblock f�r mehrere Autoren
%\begin{tabular}{lll}
%Autor 1       &Autor 2      &Autor 3 \\% Hier eintragen
% & & \\
%------------------------------------ & ------------------------------------ & ------------------------------------ \\
%\end{tabular}

%Hier unterschreiben


%%% Local Variables: 
%%% mode: latex
%%% TeX-master: "Bachelorarbeit"
%%% End: 
                 % Eidesstattliche Erklaerung Nummerierung unterdr�ckt
%%\cleardoubleemptypage         % Das Inhaltsverzeichnis auf einer rechten Seite beginnen

\pagenumbering{Roman}           % Nummerierung R�misch start bei I 

\begin{spacing}{1.0}            % Verzeichnisse werden mit einzeiligem Abstand gesetzt
 \tableofcontents               % Inhaltsverzeichnis
\end{spacing}
%%% Vorbemerkungen %%%  Nummerierung unterdr�ckt durch *
\addchap{Kurzfassung}
\label{cha:kurzfassung} 




%%% Local Variables: 
%%% mode: latex
%%% TeX-master: "Bachelorarbeit"
%%% End: 


             % Kurzfassung der Arbeit
\addchap{Abstract}
\label{cha:abtract} 







%%% Local Variables: 
%%% mode: latex
%%% TeX-master: "Bachelorarbeit"
%%% End: 
             % Abstract der Arbeit (englische Kurzfassung der Arbeit)
\addchap{Danksagung}
\label{cha:danksagung}



%%% Local Variables: 
%%% mode: latex
%%% TeX-master: "Bachelorarbeit"
%%% End: 								  % Danksagung (optional)
\addchap{Etische Aspekte}
	vieleicht hinter Grundlagen 
\label{cha:Etische Aspekte}



%%% Local Variables: 
%%% mode: latex
%%% TeX-master: "Bachelorarbeit"
%%% End: 							    % Vorwort (optional)

%%% Hauptteil %%%       Nummerierung beginnend bei 1
%%\cleardoubleplainpage         % Das erste Kapitel des Hauptteils auf einer rechten Seite beginnen
\mainmatter                     % den Hauptteil beginnen


\chapter{Einleitung}
\label{cha:einleitung}


\section {Motivation}

In der heutigen Zeit wird die Datensicherheit immer wichtiger, da immer mehr personenbezogene Daten im Internet preisgegeben werden. Um die Daten\"ubertragung zu sichern wird meistens ein asymetrisches Verschl\"usselungsverfahren verwendet.\\
Dieses Verfahren bieten zwar ein hohes Ma{\ss} an Sicherheit, hat aber auch Nachteile, wie zum Beispiel eine Erh\"ohung der Rechenzeit, die Verwaltung eines Key-Managers und die Bedrohung durch einen ,,Man-in-the-Middle'' Angriff.\\ 
Eine M\"ogliche Alternative w\"are beispielsweise die Steganographie. Dies ist die Kunst, Daten in legitimierten Datenkan\"alen zu verstecken ohne einen Verschl\"usselung anzuwenden.\\
Steganographie ist zudem sehr unauff\"allig, da in unverschl\"usselten Daten meistens keine sensiblen Daten vermutet werden.

\section {Aufgabenstellung und Zielsetzung}

Ziel der Bachelorarbeit ist die Erstellung eines Messengers, bei dem zwei Personen Daten empfangen und verschicken k\"onnen. Dabei soll die Kommunikation \"uber ein Netzwerk stattfinden und steganografisch verschl\"usselt werden.\\ Die Daten sollen nicht mit einem mathematischen Verfahren verschl\"usselt werden, sondern in ein oder mehreren Protokollen ,,versteckt'' eingebettet und \"ubertragen werden. Dazu soll ein optimales Verfahren zur Dateninfiltration und -exfiltration gefunden werden. Als Verfahren k\"onnen hier zum Beispiel Covert- Channels eingesetzt werden. Das Verfahren sollte unauff\"allig, f\"ur Dritte schwer zu interpretieren und mit gr\"o{\ss}t m\"oglicher \"Ubertragungsrate senden. Optimal w\"are ein \"ahnliche Sicherheit zu gew\"ahrleisten, wie mit einer mathematischen Verschl\"usselung.\\
Ziel ist es au{\ss}erdem jedes Dateiformat \"ubertragen zu k\"onnen. Das resultierende Programm soll in der Lage sein, gleichzeitig Server und Client zu sein, was bedeutet, dass mit dem gleichen Programm gesendet und empfangen werden kann.\\
Eine einfache GUI soll dem Benutzter das Senden und Empfangen der Daten so einfach wie m\"oglich machen.\\
W\"unschenswert w\"are ein m\"oglichst einfacher Verbindungsaufbau, der ohne den direkten Austausch der IP- Adressen statt findet.\\


%%% Local Variables: 
%%% mode: latex
%%% TeX-master: "Bachelorarbeit"
%%% End: 
               % Einleitung
%Kapitel des Hauptteils

\chapter {Grundlagen}  %Name des Kapitels
\label{cha:grundlagen} %Label des Kapitels



\section{Datenschutz}

Der Datenschutz ist ein �berbegriff f�r das in Gesetzte festgelegten Recht, dass jede Person �ber die Preisgabe der personenbezogenen Daten bestimmen kann. Die Bundesbeauftragten f�r den Datenschutz und die Informationssicherheit (BfDI) definieren den Datenschutz wie folgt:\\\\
\textit{,,Datenschutz garantiert jedem B�rger Schutz vor missbr�uchlicher Datenverarbeitung, das Recht auf informationelle Selbstbestimmung und den Schutz der Privatsph�re''}
 \cite{BfDI}\\\\
 Das bedeutet, dass jeder der personenbezogene Daten, ohne Zustimmung des Betreffenden speichert oder weiterverarbeitet vor Gericht angeklagt werden kann.
 Damit dies nicht passiert haben die meisten Institute die mit personenbezogenen Daten umgehen einen Datenschutzbeauftragten, der die Einhaltung dieser Gesetzte �berwacht.
 

\section{Datensicherheit}

Im Gegensatz zu dem Datenschutz bezieht sich die Datensicherheit nicht nur auf die personenbezogenen Daten, sondern auf alle Daten. Die Aufgabe der Datensicherheit werden durch das CIA-Prinzip beschreiben.\\ Zur Datensicherheit geh�ren nach diesem Prinzip alle Ma�nahmen, die die \textbf{C}onfidentiality, \textbf{I}ntegrity und \textbf{A}vailability (Vertraulichkeit, Integrit�t, Verf�gbarkeit) gew�hrleisten.\cite{CIA}
Eine zus�tzliche Aufgabe ist die Sicherstellung der Authentizit�t.
Viele dieser Aufgaben werden mit Hilfe der Kryptographie realisiert und umgesetzt. 


\subsection{Vertraulichkeit}
Die Vertraulichkeit ist dann gew�hrleistet, wenn die Daten nicht von unbefugten Personen eingesehen werden k�nnen. Es muss also ein System verwendet werden, bei dem sich befugte Benutzer legitimieren k�nnen und unbefugte beim Interpretieren gehindert werden. \\
In den meisten F�llen wird dies durch eine Verschl�sselung (symmetrisch oder asymmetrisch) umgesetzt. Alle legitimierten Benutzter erhalten den Schl�ssel. Die Personen ohne Schl�ssel k�nnen die Informationen nicht entschl�sseln - die Vertraulichkeit ist so garantiert.\\
Optimal w�re, wenn nicht legitime Benutzer, auch nicht an die verschl�sselten Daten kommen w�rde.

\subsection{Integrit�t} 
Die Integrit�t besch�ftigt sich damit, dass Daten nicht unbemerkt ver�ndert oder abgef�lscht werden. So soll zum Beispiel sichergestellt werden, dass eine Nachricht genau so beim Empf�nger ankommt, wie sie abgesendet wurde.\\
Hierzu k�nnen Hash-Funktionen verwendet werden, die beim Ver�ndern der Nachricht einen anderen Wert ergeben w�rden. Dabei m�sste entweder die Hash-Funktion geheim sein oder der Hash-Wert verschl�sselt werden.
 

\subsection{Verf�gbarkeit} 
Der Dritte Punkt ist die Verf�gbarkeit. Es soll immer sichergestellt wer, dass Daten aber auch Programm immer abrufbar sind. Hierzu geh�ren Mechanismen zur Vermeidung von DoS (Denial of Service) Angriffen. Diese Angriffe w�rden beispielsweise einen Server so �berfordern, dass dieser keine Dateien mehr ausliefern kann - die Verf�gbarkeit ist dann nicht mehr gew�hrleistet.

\subsection{Authentizit�t} 
Die Authentizit�t best�tigt, dass Daten von der angegeben Informationsquelle stammen. Es ist ein Identit�tsbeweis des Absenders gegen�ber dem Empf�ngers.\\ Dies kann zum Beispiel mit einer Public-Key Verschl�sselung realisiert werden. So kann der Sender die Nachrichten mit seinem Public-Key verschl�sseln und jeder im Besitz des Public-Keys kann best�tigen, dass die Nachricht genau von dieser Person kommt.

\section{Kryptographie}
Kryptographie bedeutet ,,w�rtlich: Die Lehre vom Geheimen schreiben'' \cite{hellmann2018sicherheit} und besch�ftigt sich mit der mathematischen Verschl�sselung von Informationen. Dabei gibt es zwei gro�e Verschl�sselungsarten - die Symmetrischen und die Asymmetrischen Verschl�sselungen. Bei beiden Verfahren wird durch einen Schl�ssel (meistens eine Zahl) und einem Algorithmus aus einer lesbaren Information eine Unlesbare.
Um dies wieder R�ckg�ngig zu machen wird ebenfalls ein Schl�ssel und ein Algorithmus ben�tigt.
Eine der Grundprinzipien der Kryptographie wurde bereits im 19. Jahrhundert von A.Kerkhoffs aufgestellt. 


\section{Symmetrische Verschl�sselung}

\section{Asymmetrische Verschl�sselung}

\section{Steganographie}

\section{Internet Protokolle}






 
%%% Local Variables: 
%%% mode: latex
%%% TeX-master: "Bachelorarbeit"
%%% End: 
                % Ein Kapitel des Hauptteils




\chapter{Anforderung an die L�sung}
Die Covert Channel L�sung 




\chapter{L�sungsans�tze}
\label{cha:Covert-Channel} %Label des Kapitelsne bereit

%TODO       
Im folgenden Kapitel werden verschiedene, bereits existierende steganografische Covert-Channel betrachtet, die f�r das Erreichen der Zielsetzung in Frage kommen.


\section{Covert Channel}

\subsection{Zeitabh�ngige Covert-Chanel}

Bei zeitabh�ngigen Covert-Channel werden die Daten so versendet, dass der Absendezeitpunkt oder der Abstand zwischen den Paketen die Information enth�lt. Dabei �hnelt dieses Verfahren dem in der Vergangenheit oft eingesetzten Morse Code. Hingegen beschr�nkt man sich bei diesen Covert-Channel meistens auf Bin�rdaten.
\begin{figure}[ht]
	\centering
	\includegraphics*[height=10cm]{bilder/timing.png}
	\caption{Kommunikation �ber einen Zeitabh�ngigen Kanal \cite{cabuk2004ip}}
	\label{fig5}
\end{figure}

Bei dem in Abbildung 4.1 dargestellte Covert Channel wird mit einem festen Zeitintervall gearbeitet. Dieses Zeitintervall muss sowohl dem Sender sowie dem Empf�nger bekannt sein. Wird ein Datenpaket innerhalb des Zeitintervalls gesendet wird dies als 1 interpretiert, falls kein Paket gesendet wird als 0.
So lassen sich beliebige Daten �bertragen.\cite{cabuk2004ip}

Da die Daten�bertragung sehr stark von der Netzwerkgeschwindigkeit abh�ngig ist, muss man zus�tzlich ein Verfahren zur Sicherstellung der Integrit�t implementieren.\\ 
Das Versenden des Hashwertes, der �ber einen bestimmten Anteil der Nachricht gebildet wird, k�nnte die Integrit�t garantieren.\\
Eine andere Methode bei der die Integrit�t jedoch nicht vollst�ndig garantiert, aber simpler umzusetzen ist, ist die Verwendung eines Parit�tsbit.\cite{cabuk2004ip}

\begin{figure}[ht]
	\centering
	\includegraphics*[height=10cm]{bilder/Zeitacuracyzusammenhang.png}
	\caption{Zusammenhang zwischen Zeitintervall und Genauigkeit \cite{cabuk2004ip}}
	\label{fig6}
\end{figure}

Die Daten�bertragung �ber diesen Channel ist mit einem Bit pro Paket relativ gering. Die Daten�bertragung ist hier aber auch direkt vom verwendeten Zeitintervall abh�ngig. Deshalb gilt es einen m�glichst kleinen Zeitintervall zu w�hlen und diesen optimal an die herrschenden Netzwerkbedingungen anzupassen.

Die Problematik ist hier, dass mit der Reduzierung des Netzwerkhinteralls die Fehleranf�lligkeit ebenfalls zunimmt. Abbildung 4.2 zeigt, wie das gew�hlte Zeitintervall und die Genauigkeit zusammenh�ngen. So l�sst sich bei einer langsamen Daten�bertragung mit einem Intervall von 0.05 Sekunden eine Genauigkeit von ca. 100 Prozent garantieren. Wobei diese Zahlen mit Vorsicht zu genie�en sind, da sie sehr stark vom jeweiligen Netzwerk abh�ngig sind.
Der Ersteller dieser Grafik\cite{cabuk2004ip} hat ebenfalls einen Wert $k$ in seine Implementierung eingebaut, der die L�nge einer Pause angibt, die bei einer �bertragungsverz�gerung eingelegt werden kann. Diese Verbessert zwar die Genauigkeit der Daten�bertragung die Geschwindigkeit wird aber erheblich reduziert.

Bei dieser Art von Covert-Channel kann jedes beliebige Protokoll eingesetzt werden. Es bietet sich aber an die Protokolle auf den umliegenden Netzwerkverkehr anzupassen um ihn so unauff�llig wie m�glich zu machen.


\subsection{Storage Channel}

Bei Storage Channels benutzt man Speicherattribute\cite{wendzel2012tunnel} im Protokollheadern, um unbemerkt Daten zu �bertragen. Hier wird beim Internet Protokoll angefangen, da in dieser Arbeit �ber Netzwerkgrenzen hinaus kommuniziert werden soll.


\subsubsection{IPv4}

Der IPv4 Header bietet einige M�glichkeiten Daten in Speicherattribute zu verstecken.
(Header in Kapitel Grundlagen)

\textbf{Type of Service} \\
Die letzten beiden Bits diese Feldes sind unbenutzt und k�nnen so zur Daten�bertragung verwendet werden. (2 Bits)

\textbf{Identification} \\
Bietet 16 Bits die theoretisch frei w�hlbar sind. (16 Bits)

\textbf{Reserved Flag} \\
Das erste Bit der Flags ist f�r zuk�nftige Benutzung reserviert und ist derzeit noch unbenutzt. (1 Bit)

\textbf{Fragment Offset} \\
Der Fragment Offset wird dazu verwendet, um Pakete nach einer Fragmentierung wieder zusammenzusetzen. Geht man davon aus, dass die Paket Fragmente sich frei konfigurieren lassen bietet sich die Chance 13 Bit zu verwenden. Dies ist aber fast unm�glich zu realisieren. (< 13 Bit)

\textbf{Time to Live} \\
Dieses 8 Bit Feld l�sst sich frei w�hlen. Jedoch muss man bei der Benutzung wissen, wie viele Netzwerkstationen, die dieses Feld herunterz�hlen, auf dem Weg liegen. Auch ein zu kleiner Wert kann dazu f�hren, dass die Nachricht nicht ankommt.
(< 8 Bit)

\textbf{Total Length} \\
Die Gesamtl�nge des Pakets l�sst sich auch manipulieren. Diese L�nge wird mit einem 16 Bit Wert angegeben. Jedoch ist dieser Wert durch die Mindestgr��e eingeschr�nkt. Durch Fragmentierung des Pakets oder das Hinzuf�gen von Optionen �ndert sich dieser Wert. (< 16 Bit)

\textbf{Options} \\
Dem Ip Header k�nnen Optionen hinzugef�gt werden, in die sich ebenfalls Daten einbetten lassen.

\textbf{Padding} \\
Die durch das IHL Feld angegebene Headerl�nge muss ein vielfaches von 4 Byte erreicht werden. Durch die Verwendung von Options wird diese L�nge nicht immer erreicht und wird deshalb mit Padding aufgef�llt. Dieses Padding l�sst sich theoretisch auch umwandeln und zur Daten�bertragung verwenden.

\subsubsection{IPv6}
Bei IPv6 kann man in der Regel die �quivalenten Speicherattribute verwenden, wie bei IPv4. Hier unterscheidet sich meistens nur die Namensgebung. \cite{wendzel2012tunnel}

\subsubsection{TCP}
Im TCP Protokoll bieten sich ebenfalls M�glichkeiten Daten unbemerkt zu transportieren. So kann der Source Port zur Codierung verwendet werden. Ebenfalls m�glich ist die Benutzung des optionalen TCP Timestamps, bei dem zum Beipiel die letzten Bits manipuliert werden.
\cite{wendzel2012tunnel}

Ein weiter M�glichkeit ist das Manipulieren der Sequence Number.\cite{wendzel2012tunnel} Diese Nummer gibt die Reihenfolge der Datenpakete an.
Dabei wird sie am Anfang der Daten�bertragung vom Sender errechnet und dann alle 4 Mikrosekunden um eins hochgez�hlt. Sollte diese 32 Bit Nummer �berlaufen so wird sie wieder auf null zur�ckgesetzt. So wird Sichergestellt, dass jedes Paket eine einzigartige Sequence Number bekommt\cite{TCP_RFC}\\
Um diese Nummer nun zu Ver�ndern muss eine �bersetzungsschicht eingebaut werden die, die �bertragenen Geheimdaten abf�ngt und wieder mit der Richtigen Sequence Number ersetzt.\cite{wendzel2012tunnel}


\subsubsection{Traffic Normalizers}
Storage Channels haben einen gro�en Schwachpunkt: Traffic Normalizer schreiben die oben beschriebenen Headerattribute gezielt um oder verwerfen diese wenn auff�llige Werte gesetzt sind. \\
Ein Traffic Normalizer auf IP Ebene k�nnte zum Beispiel das TTL Feld manipulieren, die Flags ,,Don't Fragment'' und ,,Reserved'' auf 0 senzen, die Options l�schen oder Pakete bei denen das IHL Feld gr��er als 5 ist verwerfen. \cite{wendzel2012problem}\\
Au�erdem denkbar ist, dass Padding und die ungenutzten Bits des Type of Service Feldes auf 0 gesetzt werden.\\
Ein solches System, in �hnlicher Weise auf alle Netzwerkschichten angewendet, ist eine sehr effektive Methode um gegen Storage Channels vorzugehen. 



\subsection{Benutzung verschiedener Protokolle}

Covert-Channel k�nnen durch die Verwendung verschiedener Protokolle realisiert werden. Hier kann man zwischen Protocol Hopping Covert Channels und Protocol Channels unterscheiden. \cite{wendzel2012tunnel}

\subsubsection{Protocol Channels}

Bei Protocol Channel werden die Daten mit Hilfe mehrere Protokolle kodiert.
Eine m�gliche Kodierung k�nnte Folgende sein:

HTTP -> 00 \tab
DNS  -> 01\\
ICMP -> 10\tab
POP  -> 11\\
 \cite{wendzel2012tunnel}

So ist man in der Lage bin�re Daten mit vier Protokolle zu versenden. In Abbildung 4.3 ist dieses Prinzip veranschaulicht.


\begin{figure}[ht]
	\centering
	\includegraphics*[height=9cm]{bilder/protoChannel.png}
	\caption{Protokol Channel \cite{wendzel2012tunnel}}
	\label{fig7}
\end{figure}

Die Wahl der Protokolle ist hier abh�ngig von den im Netzwerk verwendeten Protokollen. Nat�rlich funktioniert dieses Prinzip auch mit zwei Protokollen. Denkbar w�re hier die Verwendung von IPv4 und IPv6 da diese Protokolle unter Umst�nden unterschiedliche Wege durchs Internet nehmen und so noch unauff�lliger werden.
Wie auch bei den zeitabh�ngigen Covert-Channel betr�gt hier die �bertragungsrate 1 oder 2 Bit pro Paket. Die �bertragungsgeschwindigkeit wir durch die Netzwerkgeschwindigkeit eingeschr�nkt.\\

Eine denkbare Unterart dieses Kanals ist die Verwendung verschiedener Options im Header wodurch die Codierung realisiert wird.

\subsubsection{Protocol Hopping Covert Channels}

Dieser Kanal ist eine Mischung des Protocol-Channels und des Storage Channel. Dabei werden verschiedene Storage-Channels zu einem zusammengefasst und abwechseln Daten �bertragen.

In Abbildung 4.4 wird dargestellt wie dies realisiert werden kann. Die Bin�rdaten werden hier �ber zuf�llig gew�hlte Storage-Chanel �bertragen. Dadurch wird bewirkt, dass der Kanal unauff�lliger wird, da sich ein reales Netzwerk simulieren l�sst.

Im Beispiel unten werden jeweils 4 Bit an den Storage Channel �bergeben und an den Empf�nger weitergeleitete.

\begin{figure}[ht]
	\centering
	\includegraphics*[height=9cm]{bilder/protoHopingChan.png}
	\caption{Protokol Hoping Covert Channel \cite{wendzel2012tunnel}}
	\label{fig8}
\end{figure}

\subsection{Verwendung der Nutzdatengr��e}
Die Gr��e des Pakets l�sst sich �ber die Versendung unterschiedlicher Daten einfach manipulieren. Daraus ergeben sich etliche Varianten um einen Covert Channel zu Erschaffen. Beispielsweise kann zwischen gro�en/kleinen Paketen oder Gerade/Ungerade Datenanzahl unterschieden werden und so bin�r Daten �bertragen.\\
Aber auch direkt in die Gr��e der Nutzdaten k�nnen Informationen versteckt werden:\\
Will man 8 Bit pro Paket �bertragen, ben�tigt man die Werte von 0 bis 255. Nun muss die Gr��e der Nutzdaten so angepasst werden, dass sie jeweils den zu �bertragenden Werten entspricht. Da Nutzdaten von null oder einem Byte relativ selten sind, empfiehlt sich die Verwendung eines statischen oder flexiblen Offset der addiert wird um die Paketgr��e anzuheben.\\
Dieser Kanal l�sst sich auf alle Protokolle, die zur Daten�bertragung f�hig sind, anwenden.



\section{Steganografie}

\subsection{Klassische textbasierende Verfahren}

Botschaften in Texten teilweise einzubetten ist mit der Steganographie m�glich.
G�ngig unter Textmanipulatoren ist die gezielte Wahl des ersten Buchstaben des Wortes, wobei die Aneinanderreihung dieser Buchstaben ein neues Wort ergibt.
Bei einem wissenschaftlich erarbeiteten R�ckblick treten einige, nicht zu untersch�tzende, Sicherheitsl�cken auf, wenn diese Art der Verschl�sselung angewendet wird.

(Im oberen Text ist zur Veranschaulichung eine Nachricht an einen potentiellen Pr�fer eingebettet)

Eine weitere Methode ist die Satzzeichen zu verwenden um Informationen zu kodieren. So kann zum Beispiel ein Punkt 00, ein Komma 01, ein Fragezeichen 10 und ein Ausrufezeichen 11 bedeuten.\cite{lockwood2017text}

Durch diese Verfahren lassen sich Covert-Channel konstruieren indem ein solcher Text beispielsweise per E-Mail versendet wird.

\subsection{Basierend auf RGB Bilder}

Bildformate die auf die RGB Formate basieren sind zum Beispiel BMP (Windows Bitmap) oder auch GIF (Graphics Interchange Format). Diese Formate geben die Pixelfarbe basierend auf dem \textbf{R}ot-, \textbf{G}r�n- und \textbf{B}lauwert. In diesen Werten k�nnen Daten versteckt werden.\cite{katzenbeisser2000information} So k�nnen beispielsweise die letzten beiden Bits manipuliert werden. Diese Ver�nderung ist f�r das menschliche Auge nicht zu erkennen, da die letzten Bits kaum eine Auswirkung auf die H�he der Zahl haben.\\
Bei einer Farbtiefe von 24 Bit betr�gt die maximale �nderung der Farbwerte nur 3 Farbstufen von insgesamt 256 m�glichen.

00000000 (0) 		-> 00000011 (3)\\
11111111 (255)		-> 11111100	(252)

Auf diese Weise lassen sich 6 Bit pro Pixel �bertragen. In einem unkomprimierten HD Bild mit einer Aufl�sung von 1280x720 Pixeln (ca. 22 MByte) lassen sich 0,6912 Mbyte Daten �bertragen.

In Abbildung \ref{fig:fig4} wird dieses System veranschaulicht:

\begin{figure}[ht]
	\centering
	\includegraphics*[height=10cm]{bilder/stgRGB.jpg}
	\caption{Codierung in den 2 letzten Bits \cite{Black2018Slash}}
	\label{fig4}
\end{figure}

\subsection{Bildformate mit Alpha Kanal}
Bildformate wie PNG (Portable Network Graphics) k�nnen zus�tzlich zu den RGB Kan�len auch auch einen Alpha Kanal besitzen. Dies ist ein zus�tzlicher Wert der die Transparenz des jeweiligen Pixel angibt.
Bei einem Wert von 0 ist der Pixel ,,unsichtbar'' und bei 255 ist der Pixel komplett sichtbar.
Hier k�nnte man die Daten wie oben in den letzten beiden Werten speichern.\\
Da die Transparenz keine direkte Auswirkung auf die Farbe der Pixel hat, kann man jedoch auch mehr Daten speichern.

\subsubsection{Verwendung von ,,Shamir's Secret Sharing'' Methode}

Das in Kapitel 2.4.3 beschriebene Verfahren kann man dazu verwenden, um Nachrichten unauff�llig in den alpha Kanal eines PNG Bildes zu integrieren.\\
Die zu �bertragende Nachricht $M$ wird hierzu in Segmente von $t$ Bit, mit $t=3$ unterteilt. Bei der Umwandlung der Segmente in Dezimalzahlen entsteht so ein neues Array $M'=d_1,d_2...$ bei dem die Werte zwischen 0 und 7 liegen.

Im Gegensatz zu dem original Algorithmus greift man hier zus�tzlich auf die Werte $c_1$, $c_2$ und $c_3$ zur�ck um hier Ebenfalls ein ,,Geheimnis'' einzubetten. Zusammen mit $d$ k�nnen nun $k$ Werte integriert werden.

So ergibt sich:

$d = m_1 , c_1 = m_2 , c_2 = m_3 ,  c_3 = m_4$


Folgende Werte werden definiert:

$p = 11$\\
$x_1=1,x_2=2,x_3=3,x_4=4$

x kann hier eine beliebige Zahl annehmen und so als eine Art Verschl�sselung dienen. Fraglich ist, ob es sich dann noch um reine Steganografie handelt.

Die Werte von $q_1$ bis $q_4$ erben sich dann durch Einsetzten in sie Polynomgleichung.

$q_1 = F(x_1 ) = (m_1 + m_2 x_1 + m_3 {x_1}^2 + m_4 {x_1}^3 ) mod(p)$\\
$q_2 = F(x_2 ) = (m_1 + m_2 x_2 + m_3 {x_2}^2 + m_4 {x_2}^3 ) mod(p)$\\
$q_3 = F(x_3 ) = (m_1 + m_2 x_3 + m_3 {x_3}^2 + m_4 {x_3}^3 ) mod(p)$\\
$q_4 = F(x_4 ) = (m_1 + m_2 x_4 + m_3 {x_4}^2 + m_4 {x_4}^3 ) mod(p)$\\



Die Werte $q_1$ bis $q_4$ k�nnen nun  in den alpha Kanal eines PNG Bildes eingef�gt werden.
Durch die modulo Funktion entstehen so entstehen Werte zwischen 0 und 10. Damit das Bild eine m�glichst niedrige Transparenz bekommt muss der alpha Wert so gro� wie m�glich sein. Deshalb wird zu $q$ jeweils der Wert 245 addiert, um so nah wie m�glich an 255 zu kommen.\\
Die neu entstandenen Werte $q'_1$ bis $q'_4$ kann man nun in den alpha Kanal einf�gen und �ber eine sensible Datenverbindung �bertragen.\\


Zum Entschl�sseln muss man nun wieder 245 von den Werten des alpha Chanel subtrahieren. Durch das Erstellen und L�sen des Gleichungssystems, mit den oben gezeigten Formeln, k�nnen die Wert $m_1$ bis $m_4$ wieder berechnet werden.


\subsection{Verwenung von PDF Dateien}

Auch in PDF Dateien k�nnen Daten versteckt eingebettet werden. Das PDF Format setzt sich aus einer Reihe von Befehlen zusammen in der die Formatierung der Seite angegeben wird. So k�nnen Elemente zur Positionierung von Texten eingesetzt werden um Informationen einzubetten.\cite{zhong2007data}
Open Source Programme machen es f�r jeden m�glich auf diese Weise Daten zu verstecken.


pocgtfo -> beispiel


\chapter{Bewertung der Covert Channel}
In diesem Kapitel wird sich damit besch�ftigt, welcher der im vorhergehenden Kapitel vorgestellten Covert Channel f�r die Problemstellung geeignet ist. Es soll nun der optimale Kanal gefunden werden, der das Problem der Kommunikation zwischen Polizeipr�sidium und Informant l�sen kann.

Betrachtet man die \textbf{Textbasierenden Verfahren} etwas genauer wird schnell klar, dass diese nicht f�r gr��ere Datenmengen geeignet sind und sich auch sehr schwierig als Algorithmen darstellen lassen.

Die \textbf{Zeitabh�ngigen Verfahren} sind sehr unauff�llig da die Datenpakete nicht direkt manipuliert werden m�ssen. Es muss sich jedoch um die Integrit�t der Daten gek�mmert werden, da der Kanal sehr stark von den Netzwerkbedingungen abh�ngt. Die �bertragungsgeschwindigkeit ist mit einem Bit pro Paket ist sehr gering, jedoch l�sst sich dieser Kanal sehr gut als passiver Covert-Channel realisieren.

\textbf{Storage Channel} sind relativ auff�llig, vor allem wenn die Pakete genauer angeschaut werden. Zudem gibt es das Problem der Netzwerk Normalisierung, die den Storage Channel stark einschr�nken w�rde.
Die Methode bei der die TCP Sequence Number manipuliert wird ist hingegen f�r dieses Projekt denkbar, da sie nicht durch die Normalisierung ver�ndert werden kann und pro Paket 32 Bit �bertragen kann. Bei einer sehr genaueren Analyse kann hier aber ebenfalls auffallen, dass die Nummern nicht in der richtigen Reihenfolge versendet werden.

\textbf{Protocol Channel} sind im Gegensatz zu zeitabh�ngigen Verfahren nicht von den Netzwerkbedingungen abh�ngig und machen so eine Verfahren gegen Integrit�tsverlust �berfl�ssig.\\
Es ist schwierig einen realen Netzwerkkanal zu realisieren, da die Reihenfolge der Pakete/Protokolle zuf�llig ist. Dies ist bei realen Netzwerken nicht der Fall. So kommen zum Beispiel DNS Anfragen viel seltener vor als TCP Pakete.\\
Die Ver�nderung der Pakete ist bei diesem Covert Channel nicht n�tig. Eine implementierung als passiver Channel ist aber nicht m�glich  

\textbf{Projekt Hopping Channels} haben die gleichen negativen Eigenschaften wie die Storage Channels und kommen deshalb nicht f�r diese Projekt in frage.

Die Verwendung der \textbf{Nutzdatengr��e} f�r die Daten�bertragung ist sehr unauff�llig. Die Pakete bleiben bis auf die Nutzdaten regul�r und ziehen so fast keine Aufmerksamkeit auf sich. Der Kanal ist nicht von Netzwerkbedingungen abh�ngig und die Daten�bertragung kann 8 Bit pro Paket betragen.
Es muss jedoch eine unauff�llige Methode ausgearbeitet werden, bei der es m�glich ist variable Datenpakete zu versenden. Es ist hier ebenfalls nicht m�glich den Kanal passiv zu gestalten.

Werden \textbf{Bilddateien} zum Einbetten der Daten verwendet, muss sich �berlegt werden wie die �bertragung der Bilder stattfinden soll. Hier muss eine unauff�llige M�glichkeit gefunden werden, um diesen Austausch zu realisieren.
Hier lassen sich viele Daten auf einmal �berragen. Hingegen ist diese Methode vor allem durch die Medien sehr bekannt geworden. Ein aktuelles Beispiel ist eine Sicherheitsl�cke in Android, wo durch das �ffnen eines PNG Bildes Schadcode mit den Rechten des Benutzers ausgef�hrt werden kann.\cite{HEISE_ANDROID}.
                                  

Da der zeitabh�ngigen Covert-Channel die Pakete nicht manipulieren muss und da beim Netzwerkverkehr, bis auf den zeitlichen Offset, keine Ver�nderung vorgenommen werden muss, soll dieses Projekt mit diesem Kanal durchgef�hrt werden. Zudem ist die M�glichkeit den Channel passiv zu realisieren und die allgemeine Unbekanntheit des Channels weitere positive Aspekte.

\chapter{Umsetzung des Projekts}

\section{Konstruktion des Covert Channels}

Wie in den Grundlagen schon beschrieben, kann man einen steganografischen Kanal mit folgender Formel veranschaulichen:

\textit{Steganographischer Covert Channel = Geheime Daten + Tr�gerkanal + Steganografischer Schl�ssel}\\
 
\subsection{Geheime Daten}
Die geheimen Daten beinhalten die Information, welche versteckt �bertragen werden. Diese sollen jedes beliebige Format annehmen k�nnen. Zum Senden werden die Daten in ihre bin�ren Form �bertragen. So muss sich keine Sorge um das Datenformat gemacht werden.\\ Da ein sehr geringe �bertragungsgeschwindigkeit erwartet wird ist es hilfreich, wenn diese Daten so klein wie m�glich ausfallen.\\
Zum Entwickeln wird hierzu die Datei $test.txt$ verwendet, die als Inhalt den String ,,Hallo Welt'' besitzt.


\subsection{Steganografischer Schl�ssel}
Der steganografische Schl�ssel bildet sich aus dem, im vorhergehenden Kapitel gew�hlten, Covert Channel. Er bestimmt auf welche Art die Daten in den Tr�ger infiltriert und sp�ter exfiltriert werden.\\
Aufgrund der Wahl des zeitabh�ngigen Covert-Channel k�nnen folgende Schl�ssel definiert werden. 

Schl�ssel zur Dateneinfiltration:

\textit{Manipuliere den Datenstrom des Tr�gerkanals so, dass die geheimen Daten kodiert werden.}

Schl�ssel zur Datenexfiltration:

\textit{Lese die Zeitabst�nde der Datenpakete im Tr�gerkanal und dekodiere diese.}


\subsection{Tr�gerkanal}
Es wird ein Tr�gerkanal ben�tigt in diesen die geheimen Nachrichten eingebettet werden sollen.\\
F�r den gew�hlten Covert Channel kann prinzipiell jedes Netzwerkprotokoll verwendet werden. Da die �bertragung �ber Netzwerkgrenzen hinaus stattfinden soll muss jedoch mindestens das Internet Protokoll verwendet werden. 
Die Verwendung von ICMP Paketen ist durch die Filterung von Firewalls nicht geeignet. Die verbleibenden und sinnvollen M�glichkeiten sind demnach entweder TCP oder UDP. Da immer nur ein Bit durch ein Datenpaket �bertragen wird wird ein Datenstrom (Stream) mit vielen Datenpaketen ben�tigt.\\
Die Entscheidung Beschr�nkt sich nun auf einen UDP oder TCP Stream. UDP hat das Problem, dass es sich hierbei um eine verbindungsloses Protokoll handelt und man nicht sicher sein kann, dass die Daten in der Richtigen Reihenfolge oder �berhaupt ankommen. Durch die Verwendung eines zeitlichen Covert Channel muss aber sowieso ein System zur Sicherstellung der Integrit�t implementiert werden.\\
Auf der anderen Seite sind UDP Pakete selten geworden, da HTTP und somit die Webserver auf TCP aufbauen. Die Streams von Firmen wie Netflix oder YouTube basieren heute alle auf dem TCP/IP Stack.\\
So ist die Verwendung eines TCP Streams die Beste L�sung, wobei die verbindungsorientierte Daten�bertragung von TCP sich zus�tzlich positiv auf die Zuverl�ssigkeit des Datenkanals auswirken wird.



\section{Aufbau des Systems}

In diesem Kapitel sollen die einzelnen Bausteine, die im vorhergehenden Abschnitt beschriebenen wurden, zu einem kompletten System zusammengef�gt werden. Der Aufbau diesen Systems dient dann sp�ter als Struktur bei der Programmierung.\\
Im allgemeinen k�nnen hier zwei Systeme entstehen, entweder ein System mit einem aktiven oder passiven Covert-Channel.

\subsection{Aktiv}
Soll ein aktiver Covert Channel erstellt werden, so ist der Sender gleichzeitig als Server realisiert. Dieser sendet aktiv Datenpakete an den Empf�nger. Durch die Codierung, der geheimen Nachricht in die Zeitabst�nde zwischen den Paketen, gelangt die Information zum Empf�nger. In \textit{Abbildung 6.1} wird dieses System vereinfacht dargestellt.\\

\begin{figure}[h!]
	\centering
	\includegraphics*[height=8cm]{bilder/Activ.png}
	\caption{Aufbau eines aktiven Systems}
	\label{fig9}
\end{figure}


\subsection{Passiv}

Bei der passiven Alternative verbindet sich der Empf�nger �ber einen Proxy mit einem beliebigen Webserver, der einen konstanten Datenstrom generiert. Dies k�nnte zum Beispiel ein Video- oder Audiostream sein.\\
Dieser Datenstrom l�uft �ber den Proxy, der nun die M�glichkeit hat den Datenstrom zu manipulieren. Der Sender der geheimen Nachrichten nimmt in diesem System die Rolle des Proxys ein. Er muss die Daten nicht ver�ndern sondern nur verz�gern.

\begin{figure}[h!]
	\centering
	\includegraphics*[height=8cm]{bilder/Passiv.png}
	\caption{Aufbau eines passiven Systems}
	\label{fig10}
\end{figure}


\subsection{Bewertung des Aufbaus}
Die passive Variante bietet die M�glichkeit, die Nachrichten jedes beliebigen Servers zu verwenden und auch das Ziel zu variieren. Dies spricht f�r die passive Durchf�hrung. %TODO                                       

\section{Kodierung und Dekodierung}

Der folgende Abschnitt besch�ftigt sich damit, wie die Daten in den Tr�gerkanal codiert werden k�nnen indem nur der Zeitpunk des Versendens manipuliert wird.\\
Folgend werden zwei Methoden vorgestellt, die zur Kodierung der Bin�rdaten in frage kommen.


\subsection{Mit festen Zeitrastern}

Das in \cite{cabuk2004ip} vorgestellte Verfahren basiert auf einer Generierung von Zeitintervallen. Diese werden sowohl beim Sender und Empf�nger erstellt. Dabei haben die Intervalle des Empf�ngers einen zeitlichen Offset der ungef�hr der �bertragungsdauer entspricht. Dies dient dazu, dass Pakete die vom Sender abgesendet werden beim Empf�nger im gleichen Intervall ankommen.
Nachdem Sender und Empf�nger synchronisiert sind, kann man mit der Daten�bertragung starten.
Dabei wird ein Paket, dass in einem Zeitintervall ankommt als bin�re 1 interpretiert. Kommt kein Paket so wird eine 0 geschrieben.
Durch die Wahl l�ngerer Zeitintervalle kann die Fehleranf�lligkeit verringert werden, wobei jedoch die �bertragungsgeschwindigkeit ebenfalls abnimmt.\\
F�r diese Art der Codierung ist es unabdingbar, dass die Uhren von Sender und Empf�nger m�glichst genau �bereinstimmen.\\ Zur Veranschaulichung ist die Kodierung in \textit{Abbildung 6.3} dargestellt.

\begin{figure}[h!]
	\centering
	\includegraphics*[height=8cm]{bilder/Raster.png}
	\caption{Kodierung mit festen Zeitrastern}
	\label{fig11}
\end{figure}


\subsection{Basierend auf den Paketabst�nden}

Zur Kodierung kann ebenso die Ver�nderung der Paketabst�nde benutzt werden. Hierzu wird zwischen einer kleinen oder gro�en Pause zwischen zwei Nachrichten unterschieden. Eine gro�e Pause wird als bin�re 1 interpretiert, eine kleine Pause als 0.\\
Je nachdem wie gut die Netzwerkbedingungen sind kann hier durch eine gezielte Verkleinerung der Pausen eine Erh�hung der �bertragungsgeschwindigkeit generiert werden. Durch eine Verl�ngerung dieser Pausen sinkt jedoch die Fehleranf�lligkeit.\\
Zu sehen ist diese Art der Kodierung in \textit{Abbildung 6.4}

\begin{figure}[h!]
	\centering
	\includegraphics*[height=6cm]{bilder/Morse.png}
	\caption{Kodierung anhand der Paketabst�nden}
	\label{fig12}
\end{figure}

\subsection{Bewertung der Kodierung}
Bei der Kodierung mit Zeitrastern muss sich um die Synchronisierung gek�mmert werden. Au�erdem gilt es den Offset passend zu w�hlen. Diese Methode hat jedoch den Vorteil, dass zum Senden einer 0 keine Nachricht ben�tigt wird. Kommt es jedoch zu Fehlern im Offset oder zu pl�tzlichen �bertragungsschwankung ist dieses System sehr Anf�llig, da die Nachrichten in andere Zeitintervalle hineingeraten k�nnten.

Werden die Paketabst�nde zu Kodierung verwendet, so wird zum Senden einer 0 eine Nachricht ben�tigt. Dadurch werden mehr Nachrichten ben�tigt, die kontinuierlich zu Verf�gung sehen m�ssen.\\
Dabei besteht ein Vorteil darin, dass sich nicht um die Synchronisierung gek�mmert werden muss. Zus�tzlich l�sst sich bei schlechten Netzwerkbedingungen die �bertragung pausieren. Hierzu kann einfach die Daten�bertragung gestoppt werden. Bei der Codierung mit Rastern w�rde dieses abrupte Stoppen der Daten�bertragung als 0en interpretiert werden.
Bei einer reinen Betrachtung der Differenz zwischen den langen und kurzen Pausen ist der Sender in der Lage die Sendegeschwindigkeit beliebig anzupassen.

Aufgrund der Vorteile durch die Codierung mit Hilfe der Paketabst�nde soll diese Methode in dieser Arbeit verwendet werden. 

\section{Fehlerkorrektur}

\subsection{Parit�tsbit}
Die Fehlerkorrektur kann mit einem Parit�tsbit realisier werden. Dieses Bit gibt an, ob die Anzahl der Einsen in einer bin�ren Zahlenfolge eine gerade oder ungerade Zahl ist. Wird nun bei der Daten�bertragung ein Bit ver�ndert sich die Anzahl der Einsen und stimmt nicht mehr mit dem Parit�tsbit �berein.

Mit nur einem Parit�tsbit l�sst sich nun feststellen das ein Fehler aufgetreten ist, man erf�hrt aber nicht wo. Um auch dies herauszufinden kann der Hammingcode angewendet werden.\\
Bei dieser Codiereung werden den Datenbits mehrere Parit�tsbits hinzugef�gt. Mit Hilfe dieser Bits kann dann ein Fehler in den Datenbits gefunden werden und gezielt korrigiert werden.
\cite{HAMMING}

Die folgende Tabelle zeigt wie viele Parit�tsbits zu den Datenbits hinzugef�gt werden m�ssen um ein falsches Bit in n Datenbits zu korrigieren. 

\begin{tabular}{llllll}
Daten Bits:& 8& 16& 32& 64& 128\\
Parit�ts-Bits:& 4& 5& 6& 7& 8\\
Codewort:&12& 21& 38& 71& 136\\
	
\end{tabular}
\cite{HAMMING}

Zum verschl�sseln pr�ft jedes Parit�tsbit mehrere genau festgelegte Bits des Codeworts. Das Parit�tsbit wird dann so gesetzt, dass die Summe der gepr�ften Bits (sich selbst einge-schlossen) gerade ist.

Um einen Fehler zu erkennen wird jetzt wieder die Summe ausgerechnet. Ist die Summe gerade, so ist kein Fehler aufgetreten und die Parit�tsbits k�nnen aus dem Codeword gel�scht werden.
Ist die Summe ungerade ist ein Fehler aufgetreten. Nun m�ssen alle fehlerhaften Parit�tsbits ausfindig gemacht werden.\\
Bildet man nun die Schnittmenge aller, von nicht korrekten Parit�tsbits gepr�ften, Bits und Eliminiere alle Bits, die auch von korrekten Parit�tsbits gepr�ft wurden so bleibt das falsche Bit �brig.
\cite{HAMMING}

Da die Daten�bertragung �ber den Covert Channel auf Grund von Netzwerkschwankungen fehleranf�llig ist empfiehlt es sich den Hammingcode auf jeweils 8 Bits anzuwenden. Dies bedeutet pro Byte 4 zus�tzliche Parit�tsbits was ein Codewort von 12 Bits L�nge ergibt. In diesem Codewort kann ein potentiell falsches Bit ausfindig gemacht und korrigiert werden.

\subsection{Bewertung der Fehlerkorektur}
Da die Fehleranf�lligkeit bei schlechten Netzwerkbedingungen und hoher �bertragungsgeschwindigkeit sehr hoch ist, sollte die Fehlerkorrektur auf einen m�glichst kleine Anzahl von Bits angewendet werden. Dies hat den Nachteil, dass der Anteil von Parit�tsbits bei Verwendung von 8 Bits 50\% und bei 16 Bits 31\% ausmachen w�rden.\\
Ein weiterer Nachteil ist, dass Fehler oft nacheinander auftreten, da sie von der gleichen Netzwerkverz�gerung ausgel�st wurden. Mit dem Hammingcode kann bei mehreren Fehlern ein Fehler detektiert werden, jedoch ist eine Fehlerkorrektur nicht mehr m�glich.\\
Bei einer Verwendung von gro�en Datenpaketen und sehr verstreut auftretenden Fehlern w�rde die Implementierung des Hamming-Codes Sinn machen, aber in dieser Arbeit wird aus den oben genannten Gr�nden darauf verzichtet.


\section{Wahrung der Authenzit�t}


\section{Wahrung der Integrit�t}

Um die Integrit�t der gesendeten Daten zu gew�hrleisten, soll am Ende der Daten ein Hashwert mitgesendet werden.

Um die Integrit�t zu garantieren bildet der Sender den Hash �ber die zu Sendende Nachricht. Die Nachricht und der Hashwert werden nun versendet. Der Empf�nger bildet nun Ebenfalls den  Hash �ber die empfangene Nachricht. Stimmt dieser Hashwert mit dem �berein den er vom Sender bekommen hat, so ist die Nachricht unver�ndert versendet worden. Die Integrit�t ist sichergestellt.\\
Stimmt der Hash nicht �berein, so m�ssen die Daten erneut gesendet werden. M�glich ist auch den Hash nicht erst am Ende der Daten�bertragung zu Senden sondern nach einer festgelegten Datenmenge. Dies h�tte den Vorteil, dass nicht die komplette Nachricht wiederholt werden muss.

\subsection{Hash-Funktionen}
	\subsubsection{md5}
	Die md5 Hashfunktion wurde 1991 als Weiterentwicklung der md4 Hashfunktion ver�ffentlicht. Diese bildet eine beliebige Nachricht auf einen 128-Bit-Hashwert ab. Zu erw�hnen ist, dass der md5 aus Sicherheitsgr�nden nicht mehr empfohlen wird.\cite{watjen2018hashfunktionen}

	\subsubsection{Pearson}
	%TODO                                                                           
	Der Pearson-Algorithmus verwendet eine zuf�llig initialisierte statische Mapping-Tabelle, um jedes Byte von jedem Hash-Wert auf einen neuen Hash-Wert abzubilden.\cite{davies2010traffic}\\
	Mit diesem simplen Algorithmus lassen sich Hash-Werte mit der L�nge von einem Byte erzeugen. Die Mapping-Taballe muss jedoch auf allen beteiligten Systemen gleich sein.
	
	\subsection{Bewertung der Hash-Funktionen}
	Beide Algorithmen ver�ndern sich ma�geblich, wenn ein Bit ge�ndert wird. Der md5 ist auf Grund seiner L�nge aber auch komplexeren Algorithmus erheblich sicherer. In dieser Arbeit soll die Sicherheit jedoch allein von der Unauff�lligkeit des Covert Channel abh�ngen. Die Hashfunktion soll nur Sicherstellen, dass die Nachricht unver�ndert beim Empf�nger angekommen ist.\\
	Dazu ist der Pearson Hash ebenfalls in der Lage und hat den Vorteil, dass er nur ein sechzehntel an Datenmenge in Anspruch nimmt.\\ Aus diesen Gr�nden wird in dieser Arbeit mit dem Pearson Hash gearbeitet. 


\section{Netzwerkprotokoll}
	\subsection{Anforderung an das Netzwerkprotokol}
	Diese Anforderungen wurden bereits zum Gro�teil bei der Wahl des Tr�gerkanals formuliert. Hier wird die Nutzung eines TCP basierenden Protokolls festgelegt.\\
	Zus�tzlich kommt nun der Aspekt hinzu, dass sich mit diesem Protokoll legitim gro�e Datenmengen verschicken lassen sollen ohne Aufmerksamkeit zu erregen. Um der Philosophie von Covert Channels und der Steganographie gerecht zu werden soll ebenfalls keine Verschl�sselung eingesetzt werden.
	
	\subsection{HTTP}
	Das Hypertext Transfer Protocol (HTTP) ist ein Protokoll auf Anwendungsebene.
	Es ist ein generisches, zustandsloses Protokoll, das f�r viele Aufgaben verwendet werden kann, die auch �ber die Verwendung f�r Hypertext hinausgehen.
	HTTP wird zur Versendung von Webseiten und Informationen seit dem Jahr 1990 verwendet.\cite{fielding1999hypertext}\\
	Auch heute stellt es zusammen mit der verschl�sselten Variante HTTPS einen elementaren Bestandteil des Internets dar. Dabei besteht die Hauptaufgabe darin Daten von Webservern in den Browser zu laden.
	\subsection{SMTP}
	Das Simple Mail Transfer Protocol (SMTP) hat die Aufgabe Mail zuverl�ssig und effizient weiterzuleiten. SMTP Nachrichten, und die in ihnen enthaltenen Mails werden von SMTP Servern entgegengenommen um sie an den Empf�nger weiterzuleiten. Auch die Kommunikation zwischen den SMTP Servern wird mit Hilfe dieses Protokolls realisiert.\cite{klensin2001rfc}
	\subsection{FTP}
	Mit dem File Transfer Protocol (FTP) k�nnen Dateien �ber ein Netzwerk zu einem Server hoch- und heruntergeladen werden. Ebenfalls ist es m�glich das Dateisystem auszulesen und auf entfernten Rechnern Dateien zu erstellen aber auch zu l�schen.
	Die hierzu n�tige Kommunikation, wird durch das FTP definiert.\cite{postel1985rfc}
	
	\subsection{Bewertung des Netztwerkprotokolls}
	Alle Protokolle basieren auf TCP und verzichten auf eine Verschl�sselung. FTP wird heute in der Regeln aber nicht mehr angewendet, da es von Protokollen wie SFTP und SSH abgel�st wurde. Daher w�re eine Verwendung.\\
	Auch bei SMTP wird heute vermehrt auf die SSL verschl�sselte Variante zur�ckgegriffen. Ein weiteres Problem be SMTP ist, dass die Nachrichten bei gro�en Paketmengen als Spam interpretiert werden kann.\\
	HTTP ist trotz der Einf�hrung von HTTPS im Internet immer noch weit verbreitet und zieht sehr wenig Aufmerksamkeit auf sich. Zus�tzlich l�sst sich durch die Bereitstellung einer Webseite die wahre Aufgabe des Webservers verbergen.\\
	Aus diesen Gr�nden soll in dieser Arbeit ein HTTP Server realisiert werden, der als Sender der geheimen Daten dient.

\section{Server}
	\subsection{Anforderungen an den Server}
	Der Server muss in der Lage sein, einen HTTP Request entgegenzunehmen und im Gegenzug eine Webseite ausliefern. Da die geheimen Daten vom Server an den Client �bertragen werden sollen, muss der Server in der Lage sein aktiv ohne Requests HTTP Nachrichten an den CLient zu senden.\\ Zus�tzlich ist es f�r diese Anwendung essentiell, dass sich das Senden der Nachrichten verz�gern l�sst um die n�tigen zeitlichen Abst�nde f�r den Covert Channel zu realisieren. Aus diesem Grund soll der Server frei programmierbar sein und auch Daten verarbeiten k�nnen um beispielsweise bin�r Daten zu erstellen oder einen Hashwert zu generieren.\\
	Der Server soll leichtgewichtig, einfach zu bedienen und auf einem Linux Betriebssystem lauff�hig sein.
	
	\subsection{Java HttpServer}
	HttpServer ist eine Java Klasse die es erm�glicht einen einfachen HTTP Server zu erstellen.
	Die von Oracle angebotene Klasse implementiert einen Webserver der an eine IP-Adresse und an einen Port gebunden ist und dort auf eingehende TCP Verbindungen lauscht.\\
	Um den Server nutzen zu k�nnen m�ssen im ein oder mehrere HttpHandler hinzugef�gt werden. Diese bearbeiten, die Anfragen auf verschiedene URL Pfade.\\
	Der HttpServer kann bei der Verwendung der Unterklasse HttpsServer auch verschl�sselte Verbindungen realisieren.\cite{JAVA_HTTP}
	
		
	\subsection{Node.js und Express}
	Node.js ist eine Plattform, ausgerichtet um Netzwerkanwendungen zu erstellen. Dabei ist Node.js eine asynchrone und ereignisgesteuerte JavaScript Runtime. Node.js wird in JavaScript programmiert und kann mit Paketen von npm (Node Paket Manager) erweitert werden.\\
	Als asynchrone Laufzeitumgebung arbeitet Node.js sehr viel mit Callback-Funktionen, die bei Erf�llen von Events ausgef�hrt werden. Durch die asynchrone Abarbeitung des Programmcodes entsteht eine sehr gut skalierbare Anwendung die keine Deadlocks generieren kann.\cite{NODE}\\
	Um mit Node.js ein Webserver zu erstellen kann das Web-Framework Express verwendet werden. Express l�sst sich mit Hilfen von npm in das Projekt eingliedern. Das Express Objekt ist in der Lage auf einem Port auf TCP Verbindungen zu warten und entgegenzunehmen. Jedem Pfad ist eine Callback-Funktion zugeordnet, die bei dessen Aufrufen die Abarbeitung der Anfrage �bernimmt.\cite{EXPRESS}
	
	
	\subsection{Bewertung des Servers}
	Da der Server in der Lage sein muss HTTP Nachrichten zeitlich verz�gert abzuschicken, sind fertige und schwergewichtige Server wie der Apache oder NGINX ungeeignet.\\
	Bei beiden oben vorgestellten Servern hat man die M�glichkeit den Datenfluss zu manipulieren und so auch zu verz�gern. Da es sich um normalen Java oder JavaScript Code handelt, kann man beliebige Funktionen eigenh�ndig implementieren und auch von Dateien lesen. So ist der Java HttpServer und auch Express in der Lage den Covert Channel zu realisieren.\\
	Der sehr simple Aufbau und das einfache Hinzuf�gen und Verwenden von Paketen sprechen jedoch f�r den Node.js Server, weshalb dieser hier verwendet wird. Ein weiterer Vorteil von Node.js ist das einfache Installieren auf einem Linux System und dass keine zus�tzliche IDE ben�tigt wird.
	
	\section{Back-End}
		\subsection{Express Implementierung}
			Das importierte Express Paket wird als \textit{app} Objekt in den Code eingebunden. Mit Hilfe dieses Objekts kann definiert werden, wie auf einen Request an eine definierte URL-Routen reagiert werden sollen.\\
			 Eine Route wird wie in Listing 6.1 gezeigt hinzugef�gt. Die mitgegebene Callback-Funktion wird ausgef�hrt, falls ein GET Request an die ,,Wurzel-Route'' erfolgt. Ist dies der Fall so wird hier die index.html Seite ausgeliefert.\\
				
			\begin{lstlisting}[caption=Hinzuf�gen der Stamm-Route]
				app.get('/', function(req, res){
				res.sendFile(__dirname + '/index.html');
				});
				
			\end{lstlisting}
				
				
			Ebenfalls von Express stammt das \textit{http} Objekt, welches den Http Server darstellt. Der Server Port kann wie in Listing 6.2 gew�hlt werden.\\
			

			\begin{lstlisting}[caption=W�hlen des Server Ports]
				http.listen(80, function(){
					console.log('listening on port:80');
				});
					
			\end{lstlisting}
			
			\subsection{Kommunikation des Covert Channel}
				Bei der Express Implementierung wird die \textit{index.html} Datei mit Hilfe von REST (Representational State Transfer) ausgeliefert. REST ist ein Architekturstil, der die Kommunikation zwischen Server und Client regelt. Dieses System beruht darauf, dass ein Server eine beliebige Ressource, wie beispielsweise \textit{index.html}, anbietet. Die wichtigsten Methoden um mit diesen Ressourcen umzugehen sind  GET, PUT, POST und  DELETE.\cite{barton2010modellierung}\\
				Diese Methoden, auf die mit http zugegriffen wird, sehen nicht vor, dass ein Server selbst�ndig und ohne danach gefragt zu werden Nachrichten an den Client sendet.

				\subsubsection{Anforderung an die Kommunikation des Covert Channels}
				Da die Hauptaufgabe des Server das Senden manipulierter Pakete an den Client sein wird, muss ein alternative Kommunikation gefunden werden, die nach dem Ausliefern der Webseite den Nachrichtentransport �bernimmt. Dabei muss eine bidirektionale Kommunikation m�glich sein. Auch eine einfache Anwendung soll angestrebt werden.
					
				\subsubsection{Websockets}
				
				Das Websocket Protokol baut auf HTTP auf. Bei HTTP wird, nach dem der Client die Antwort vom Server erhalten, hat meistens die Verbindung geschlossen. Bei der Verwendung von websockets wird die darunterliegende TCP/IP Verbindung weiterverwendet und kann wie ein normaler Netzwerksocket benutzt werden. Dies erm�glicht dem Client und dem Server jeder Zeit Daten zu senden.\cite{abts2015bidirektionale}
					
				\subsubsection{Socket.IO}
				
				Socket.IO ist eine JavaScript-Bibliothek, mit der sich eine bidirektionale Echtzeitkommunikation realisieren l�sst. Die Funktion �hnelt einem WebSocket und l�sst eine ereignisgesteuerte Kommunikation zwischen Browser und Server zu.\cite{SOCKET_IO}\\
				Um die Verbindung zwischen Server und Client auf jeden Fall sicherzustellen und um so viele Browser wie m�glich zu unterst�tzten, setzt Socket.io auf mehrere Technologien, wie Websockets, Flash-Sockets oder Comet.\cite{ullrich2012websockets}
				
				
				\subsubsection{Berwertung der Kommunikation des Covert Channels}
				Beide Methoden sind in der Lage den Covert Channel zu realisieren, jedoch gibt es bei Websockets Probdenleme mit der Verwendung von Proxys.\\
				Socket.IO stellt eine Erweiterung der WebSockets dar und verf�gt �ber weitere Funktionen wie zum Beispiel Broadcasting. Aus diesen Gr�nden, und da eine sehr komfortable node.js Schnittstelle vorhanden ist soll hier die Kommunikation mit Socket.IO realisiert werden.
				
			
			\subsection{Socket.IO Implementierung}
			Socket.IO wird in Form des \textit{io} Objekts in den Code eingebunden. Der Verbindungsaufbau ist wie in Listing 6.3 gezeigt aufgebaut. Den Events werden ihre jeweiligen Callback-Funktionen zugeordnet.\\
			Wurde eine Verbindung aufgebaut so sendet der Server eine \textit{test} Nachricht an den Client. Hat der diese empfangen so antwortet dieser mit einem \textit{ClientHello}. Wird diese vom Server empfangen, so wird der jeweilige Socket in ein Array gespeichert, wo er sp�ter zum Senden von weiteren Daten verwendet werden kann.\\
			
			
			\begin{lstlisting}[caption=Verbindungsaufbau mit Socket.IO]
				io.on('connection', (socket) => {
					var address = socket.handshake.address.replace(/^.*:/, '');
					timestaps.push({ip: address ,time: getMinute()});
				
					socket.emit('test', 'test');
				
					socket.on('ClientHello', function (data) {
						console.log("Client connected");
						socken.push(socket);
					});
				
			\end{lstlisting}
				
		
		\subsection{Geheime Daten}
		Die geheimen Daten werden vom Dateisystem gelesen und anschlie�end mit Hilfe des npm Paketes \textit{buffer-bits} in einen Bit-String umgewandelt.
				
		\subsection{Pearson Hash Implementierung}
		Um die Pearson Hash-Funktion auf die Daten anzuwenden, muss eine Mapping-Tabelle mit den Werten von 0-255 in zuf�lliger Reihenfolge generiert werden. Diese l�sst sich beispielsweise mit einem Python Script unter Verwendung der \textit{shuffel} Funktion erstellen.\\
		Die hier generierte Tabelle muss beim Server sowie Client bekannt sein, um bei gleicher Hash-Funktion und gleichen Daten den identischen Hashwert zu generieren.\\
		Den Algorithmus zur Berechnung des Hashes ist in Listing 6.4 gezeigt.
		Hier wird am Anfang ein Hash-Wert \textit{h} generiert. Dieser wird bitweise XOR mit dem Datenwert verkn�pft. Der entstandene Wert wird als Index in der Mapping-Tabelle verwendet. Der dortige Wert in der Mapping-Tabelle wird der neue Wert \textit{h}. Diese wird so lange wiederholt bis alle Datenwerte verwendet wurden. 
		
		\begin{lstlisting}[caption=Erstellen der Mapping-Table]
					
					for (var j = 0; j < hashLength; j++){  
						var h = table[(parseInt(data.charAt(0)) + j) % 256];
						for (var i = 1; i < data.length; i++){
							h = tabel[(h ^ data[i])];          // XOR
						}
						hash[j] = h;
					}
		
		\end{lstlisting}
		
		Da hier mit einer \textit{hashLength} von 1 gearbeitet wird, ist das Ergebnis ein 8 Bit Wert der nun an die Daten angeh�ngt werden kann.
		
		\subsection{Covert Channel Implementierung}
		
		Um den Covert Channel zu implementieren werden Nachrichten zeitlich verz�gert an den Client gesendet. Grunds�tzlich kann jede beliebige Nachricht an den Client gesendet werden. Hier wird zum Testen die aktuelle Uhrzeit versendet.\\
		Die L�nge der Pause zwischen den Daten h�ngt davon ab, ob eine 1 oder 0 �bertragen werden soll. Bei einer 1 wird eine lange Pause zwischen den Paketen gemacht bei einer 0 eine kurze.\\
		Zur Definition der Pausen wird die lange Pause in Millisekunden angegeben. Zus�tzlich kommt ein Faktor der das Verh�ltnis zwischen der langen und kurzen Pause angibt. Durch die Variation dieser beiden Werte kann der Covert Channel eingestellt und optimiert werden.
		Ist eine komplette Datei �bertragen wird eine definierte Pause von einer Sekunde gemacht. Danach beginnt die Daten�bertragung erneut.
		
		\begin{lstlisting}[caption=Covert Channel]
		
					async function covertChannel(){
						while (true) {
							if (fileLoad == true) {
								for(i = 0; i < dataBits.length; i++){
									if(socken.length != 0){
										socken[0].emit('time', getTimeString());
									}
									if(dataBits[i] == "1"){
										await sleep(longBreak);
									}
									else {
										await sleep(shortBreak);
									}
								}
								if(socken.length != 0){
									socken[0].emit('time', getTimeString());
								}
							}
							await sleep(breakBetweenTransmit);
						}
					}
		
		\end{lstlisting}
	\section{Front-End}
		\subsection{HTML}
		
		Das Front-End wird durch eine einfache HTML-Seite ausgeliefert. In diesem Proof-of-Concept Projekt wird als Funktion der Webseite das Anzeigen der aktuelle Uhrzeit umgesetzt. Diese Uhrzeit wird vom Server empfangen und danach im Browser angezeigt.\\
		
		\subsection{JQuery}
		
		JQuery ist eine JavaScript Bibliothek, die es wesentlich einfacher macht das HTML-Dokument und den DOM-Baum zu manipulieren und zu ver�ndern. Hier wird es dazu verwendet, um jeweils die aktuelle Uhrzeit dynamisch auf der Seite anzuzeigen.\\
		Zudem kommt es zum Einsatz, um die Eventhandler f�r Socket.IO einzubinden.
		
		
		\subsection{Socket.IO}
		Der n�tige Code wird f�r das Paket wird �ber ein Script Tag heruntergeladen und hinzugef�gt. Wie auch serverseitig kann hier ein ServerIO Objekt generiert werden.
		Diesem werden hier die Events zum Empfangen der Nachrichten zum Verbindungsaufbau und zum Empfangen der Zeitpakete hinzugef�gt.\\
		
		
	\section{Clientseitige Auswertung des Covert Channel}
		Die vom Server erhaltenen Pakete und vor allem die Abst�nde zwischen den Paketen m�ssen nun ausgewertet werden, um die geheime Nachricht zu rekonstruieren.
		
		\subsection{Anforderung an die Auswertung}
		F�r die Auswertung muss eine Anwendung geschrieben werden, die die eingehenden Nachrichten vom Server analysiert. Dazu muss der exakte Zeitpunk des Eintreffens der Pakete so genau wie m�glich aufgezeichnet werden. Nach dem Rekonstruieren der geheimen Nachricht muss diese auf das Dateisystem abgespeichert werden. Ein weiterer wichtiger Aspekt ist, dass es f�r Dritte m�glichst schwierig sein soll die Analyse des Covert Channels zu bemerken.
		
		\subsection{Auswertung im Front-End}
		Bei der Auswertung im Front-End werden die Daten direkt nach dem Empfangen im Browser ausgewertet werden. Hier sind die n�tigen Funktionen mit JavaScript geschrieben und direkt im Browser ausgef�hrt. Sobald die Nachricht beim Browser ankommt wird die Event Callback- Funktion ausgef�hrt wo dann der aktuelle Zeitstempel abgespeichert werden kann.
		
		\subsection{Auswertung mit externem Programm}
		Bei einer Auswertung mit einem externen Programm hat eine zweite Anwendung, die unabh�ngig vom Browser ist, die Aufgabe die eingehenden Nachrichten zu interpretieren. Dazu muss die Anwendung in der Lage sein, den Netzwerkverkehr mitzulesen und den Zeitpunkt des Eintreffens abzuspeichern und zu interpretieren.
		
		\subsection{Bewertung der Covert Channel Auswertung}
		Die Auswertung direkt im Front-End hat den Vorteil, dass kein zweites Programm ben�tigt wird. So ist diese Variante ressourcenschonender und vereinfacht die tats�chliche Anwendung.\\
		Die Auswertung direkt im Front-End hat jedoch einen gro�en Nachteil: Der Front-End Code wird an jeden versendet, der die Seite aufruft. So kommen Dritte, die m�glicherweise die Kommunikation abh�ren an den Sourcecode, der den Covert-Channel auswertet. Zus�tzlich w�re dadurch der Algorithmus der Hash-Funktion bekannt und auch die zugeh�rige Mapping-Tabelle. Dies w�rde einem Dritten, der als Man-in-the-Middle zwischen dem Server und Client steht, die M�glichkeit geben selbst Nachrichten zu verfassen oder diese zu manipulieren.\\
		Aus diesen Gr�nden wird die Auswertung des Covert Channel durch ein externes Programm realisiert.
	
\section{Client}
	Der Client stellt hier den Empf�nger des Covert Channels dar. Wie im vorhergehenden Kapitel festgelegt, soll dieser als externes Programm realisiert werden, dass unabh�ngig vom Browser.
	
	\subsection{Programmiersprache}
		\subsubsection{Anforderung an die Programmiersprache}
			Die zu verwendende Programmiersprache muss in der Lage sein Daten effizient zu verarbeiten. Eine harte Echtzeit ist jedoch nicht n�tig. Es soll ein Programm entstehen, dass auf allen Unix Systemen lauff�hig ist und das sich �ber die Konsole oder ein Shell-Script �ffnen l�sst.\\
			Es soll m�glichst einfach sein ein anderes Konsolenprogramm zu �ffnen und dessen Output zu empfangen und auszuwerten.
			
		\subsubsection{Java}
			Java ist eine objektorientierte Programmiersprache dessen Code auf mehr als 3 Milliarden Ger�ten ausgef�hrt wird. Der Java Code orientiert sich an C++ aber auch an anderen Skript Sprachen.\\
			Der Java Code wird von einem Compiler in Bytecode umgewandelt. Dieser kann nun auf jedem Ger�t das die Java Runtime besitzt ausgef�hrt werden. Dadurch muss der Code nicht mehr f�r jedes Ger�t compiliert werden.\cite{ullenboom2004java}
			
		\subsubsection{Python}
			Python ist eine objektorientierte Skript Sprache. Der Code wird als lesbares Skript an den Anwender �bergeben und von einem Python Interpreter ausgewertet. So wird kein Compiler und auch nicht unbedingt eine IDE ben�tigt.
			Die Python Syntax ist so entworfen, dass die Skripte sehr gut lesbar und auch wiederverwendbar sind.\\
			Trotzdem ist der Code sehr kompakt und hat in der Regel ein Drittel bis ein F�nftel der Codel�nge von traditionellen Programmiersprachen wie Java oder C++.\cite{weigend2006objektorientierte}
		\subsubsection{Bewertung der Programmiersprache}
			Beide Programmiersprachen sind sind in der Lage das Clientprogramm zu realisieren. Java zeigt im direkten Vergleich mit Python eine bessere Performance. Python hingegen ist besser daf�r geeignet um mit zus�tzlichen Konsolenprogrammen zu arbeiten und bringt von Haus aus viele Funktionen mit die bei der Datenverarbeitung helfen. Da es sich hier nicht um ein komerzielles Projekt handelt, kann das Skript einfach verbreitet werden und nach Bedarf angepasst werden.
			Aus diesen Gr�nden wird in diesem Projekt Python als Programmiersprache verwendet.
	
	
	
	\subsection{Mitschneiden der Datenenpakete}
		Paket-Sniffer sind Programme, die in der Lage sind den Netzwerkverkehr auf den verschieden Interfaces aufzuzeichnen und f�r den Benutzter zu veranschaulichen. Ein solches Programm soll hier verwendet werden um die Datenpakete mitzuschneiden.
	
		\subsubsection{Anforderungen an den Paket-Sniffer}
		Der Paket-Sniffer soll sich �ber die Konsole �ffnen lassen und die Ergebnisse in eine Pipe schreiben. Das Tool soll frei f�r alle UNIX Systeme erh�ltlich. Die Pakete sollen m�glichst korrekt mitgeschnitten werden und der Zeitpunk des Eintreffens des Pakets soll ausgegeben werden.\\
		Zus�tzlich soll das Filtern der Pakete m�glich sein um die mitgeschnittenen Pakete auf die vom Server einzugrenzen.
		
		\subsubsection{tcpdump}
		tcpdump ist eine Konsolenanwendung f�r UNIX Systeme, die die empfangenen Netzwerkpakete an einer Netzwerkschnittstelle ausgibt. Dabei basiert tcpdump auf der Betriebssystemschnittstelle \textit{libpcap}. Es lassen sich ebenfalls Filter einstellen die beispielsweise nur Pakete von einem bestimmten Host aufzeichnen. \cite{jacobson2003tcpdump} 

		\subsubsection{tshark}
		tshark ist eine Version von Wireshark, die dessen volle Funktion in der Konsole aufrufbar macht. tshark ist ein sehr m�chtiges Tool um Netzwerkpakete nicht nur aufzuzeichnen, sondern auch zu Decodieren. Es lassen sich ebenfalls unz�hlige Filter realisieren.\cite{TSHARK}

		\subsubsection{Bewertung der Paket-Sniffer}
		tshark hat einen gr��ern Funktionsumfang. In diesem Projekt wird aber nur der Zeitstempel der Nachricht ben�tigt. tcpdump ist hierzu in der Lage und hat ebenfalls den Vorteil, dass er bei vielen UNIX Systemen wie zum Beispiel Ubuntu bereits vorhanden ist und nicht installiert werden muss. 
		
		\subsubsection{Implementierung}
		F�r das �ffnen und anschlie�ende Mitschneiden der Daten wird ein eigener Thread gestartet. Hier wird zuerst mit \textit{os.popen} ein neuer Prozess erstellt in dem tcpdump l�uft. Die n�tigen Parameter f�r die Einstellung des Filters werden ebenfalls �bergeben. Der R�ckgabewert ist hier ein \textit{open file object}, in welches die Pipe zu tcpdump die Ergebnisse schreibt.
		Aus diesem \textit{open file object} kann nun die tcpdump Ausgabe gelesen und der Empfangszeitpunkt in einen Buffer gespeichert werden.\\
		Die Zeitstempel werden dann kontinuierlich in ein globales Datenarray geschrieben. Um Kollisionen zu vermeiden, wird ein Mutex verwendet.
		
		\begin{lstlisting}[caption=Paket-Sniffing]
			   	pipe = os.popen("tcpdump -s 0 host "+host+" and src port "+port+" -q -i any -l")
					for line in pipe:
						buffer.append(line[0:15])
						
						if len(buffer) > bufferzize:
							mutex.acquire()
							data = data + buffer
							buffer = []
							mutex.release()
		\end{lstlisting}
		
	\subsection{Interpretieren der Zeitstempel}
		Die Zeitstempel vom tcpdump Thread sollen nun ausgewertet. Daf�r werden die als String abgespeicherten Zeitstempel in \textit{datetime} Objekte geparst, um mit ihnen Rechnen zu k�nnen.\\
		Die hiermit berechneten zeitlichen Diverenzen werden entweder als eine eine bin�re 1 oder 0 interpretiert. Wie auch schon serverseitig wird die Geschwindigkeit des Covert Channels durch die Angabe der langen Pause und dem Faktor, der den Unterschied zwischen der langen und der kurzen Pause beschreibt definiert.\\
		So kann zum Beispiel eine lange Pause als 50 Millisekunden und der Faktor als 0.5 definiert werden. Daraus ergibt sich eine eine kurze Pause von 20 Millisekunden.
	
	\subsubsection{Toleranz}
		Da die Zeiten mit sechs Nachkommastellen angegeben werden ist es unm�glich, dass die Pausen exakt der Angabe entsprechen. Durch Verz�gerungen durch Schwankungen in der Netzwerkgeschwindigkeit aber auch durch Prozess-Scheduling kann es zu Abweichungen kommen.\\
		Deshalb muss die Pause nicht als festen Zeitpunkt, sondern als Zeitfenster definiert werden. Ist die L�nge der Differenz zwischen den Paketen im jeweiligen Zeitfenster enthalten, kann sie entsprechend interpretiert werden. Falls nicht wird diese Paket ignoriert.\\
		Die Zeitfenster werden durch die prozentuale Angabe, bez�glich der Pausenl�nge, in Positive und Negative Richtung angegeben.\\
		Wird die Pause mit 50 Millisekunden, einer Positiven Toleranz von 30\% und einer Negativen Toleranz von 10\% angegeben, ergibt sich ein Zeitfenster zwischen 45 und 65 Millisekunden.\\
		Da nur die Differenz der Nachrichten betrachtet wird k�nnen Nachrichten auch zu fr�h kommen, wenn die vorhergehende Nachricht erheblich zu sp�t ist. Aus diesem Grund muss dass Zeitfenster auch  in Negative Richtung erweitert werden.\\
		Von der Einstellung dieser Toleranzfenstern h�ngt erheblich die Qualit�t der empfangenen Daten ab.\\
		Der Code zeigt, wie eine Klassifizierung der Differenz, hier \textit{f1} genannt, in die Zeitfenster realisiert ist.
	
		
		\begin{lstlisting}[caption=Interpretation mit Zeitfenstern]
			if write == True:
				if sBigBreakTolerance < f1 < bBigBreakTolerance:   
					codedata.append("1")
					print(str(f1) + "  \t=> 1 ")    
				else:
					if sSmallBreakTolerance < f1 < bSmallBreakTolerance: 
						codedata.append("0")
						print(str(f1) + "  \t=> 0")
					else:
						print(str(f1) + "  \t=> undefind: will be ignored")
		\end{lstlisting}
		
		
	\subsection{Verarbeiten der Erhaltenen Daten}
		Von den erhaltenen Daten werden die letzten 8 Bit abgeschnitten. Dies ist der Hash-Wert vom Server. Um die Daten zu validieren wird auf diese die gleiche Hash-Funktion angewendet. Stimmt dieser Hash-Wert mit dem vom Server �berein, so wird die Datei ins Dateisystem geschrieben.\\
		Damit die Daten nicht codiert auf die Datei geschrieben werden, wird das Paket \textit{BitArray} verwendet, welche es m�glich macht die bin�ren Daten zu schreiben.\\
		Die Datei oder Nachricht ist hiermit erfolgreich �bertragen.
		
\section{Optimales Einstellen des Covert Channel} 
	\subsection{Messen der durchnittlichen Abweichung}
		in prozent
	
\chapter{Evaluation}


\chapter{Umsetzung der passiven Variante}
	Die in Abbildung 6.2 dargestellte passive L�sungsvariante des Covert Channels soll ebenfalls betrachtet werden. Dies hat den Vorteil, dass als Server jeder beliebige Webserver verwendet werden kann. Die Codierung wird nun von einem Proxy realisiert.\\
	F�r die Proof of Concept Implementierung wird hier der Server und der Client der aktiven Variante wiederverwendet. Im Browser des Clients muss die Verwendung eines Proxys eingetragen werden.

	\subsection{Anforderung an den Proxy}
		Der Proxy hat die Aufgabe TCP Nachrichten von einem Client an einen Server zu senden. Die Nachrichten sollen bidirektional durch den Proxy geschleust werden. Nachrichten vom Client werden an den Server weitergeleitet und Nachrichten vom Server an den Client. Dabei soll der Inhalt der Nachrichten nicht ver�ndert werden. Wie bei der aktiven Variante soll der Proxy in der Lage sein die TCP Pakete zeitlich zu verz�gern um so Daten zu �bertragen.\\
		Der Proxy soll auf einem Linux System laufen.
	
	\subsection{L�sungsans�tze}
		\subsubsection{Ver�nderung eines Open-Source Proxys}
			Ein L�sungsansatz ist die Verwendung eines fertigen Proxys wie zum Beispiel Squid oder Tinyproxy. F�r diese Open Source Programme ist der Code frei erh�ltlich. Hier gilt es die Funktion zum Senden der Nachrichten so zu ver�ndern, dass die Nachrichten verz�gert werden.
		
		\subsubsection{Proxy selbst programmieren}
			Eine alternative L�sung ist die eigenst�ndige Programmierung eines Proxys. Dazu soll ein Python-Skript geschrieben werden, das jeweils ein Netzwerksocket zum Client und einen zum Server verwaltet. Werden die TCP Pakete vom einen an den anderen Socket weitergegeben erh�lt man einen Proxy. Die Pakete k�nnen hier frei ver�ndert werden.
	
	\subsection{Bewertung des L�sungsansatzes}
			Beide L�sungsans�tze sind in der Lage die Anforderungen umzusetzen. Der selbst programmierte bietet jedoch die volle Kontrolle und bietet bei dieser Proof of Concept Implementierung mehr Spielraum. Zudem ist diese sehr leichtgewichtige Software Variante komfortabler zu Debuggen und optimieren. Aus diesen Gr�nden soll hier der Proxy eigenh�ndig erstellt werden.\\
			Sollte sich der Proof of Concept als Erfolg herausstelle, ist die erste L�sung f�r eine reale Anwendung besser geeignet, da so eine gro�e Funktionsumfang hinzukommt.
	
	\subsection{Implementierung}
		\subsubsection{Sockets}
			Es wird ein Listen-Socket erstellt, der auf eingehende Nachrichten auf Port \textit{8080} h�rt. Dieser Port muss sp�ter in den Proxy-Einstellungen des Browsers hinterlegt werden.\\
			Sendet der Browser einen HTTP-Request an den Proxy, wird der Client-Socket erstellt. Der Proxy erstellt nun den Server-Socket indem er sich mit dem in dem HTTP-Request stehenden Webserver verbindet. Da oft externe Ressourcen verwendet werden und Verbindungen nach dem Senden einer Datei geschlossen werden, entstehen pro Webseite mehrere dieser Verbindungen.\\
			Um die hierdurch entstehenden Sockets zu verwalten wird das Betriebssystemmodul \textit{select} verwendet. Bei der Verwendung von \textit{select} wird das Betriebssystem gefragt ob die angegebenen Sockets bereit zum Lesen oder zum Schreiben sind. Ist dies der Fall wird eine Liste mit beschreibbaren und eine mit lesbaren Sockets zur�ckgegeben. Dies hat den Vorteil, dass nicht alle Sockets mit Polling abgefragt werden m�ssen.\\
			In Listing 7.1 ist der Code zu sehen, der die Handhabung der Sockets �bernimmt. Die Liste aller Sockets wird zur Kontrolle an \textit{select} �bergeben.
			
			\begin{lstlisting}[caption=Socket Handling]
while True:
	readable, writable, exceptional = select.select(self.lsock, self.lsock, self.lsock)
	
	self.read(readable)
	self.send(writable)
			\end{lstlisting}
			 
			 Anhand ihres Index in der Socket-Liste ,k�nnen die Sockets identifiziert werden. Dabei geh�ren immer ein Client- und ein Server-Socket zusammen. Die Nachrichten die zwischen diesen zwei augetauscht werden sollen, sind in einem Array gespeichert.\\
			
		\subsubsection{Geheime Daten}
			Wie auch schon bei der aktiven Variante, werden die zu �bertragenden Daten von einer angegebenen Datei gelesen und in eine bin�re Liste umgewandelt. Hieraus wird ein 8 Bit Pearson-Hash berechnet und hinten angeh�ngt.
			
		\subsubsection{Covert Channel}
			Bei der Erstellung des Covert Channels m�ssen die Nachrichten an den Client nun zeitlich Verz�gert werden. Dabei besteht allerdings das Problem, dass bei dieser Anwendung nur ein Thread verwendet wird, der gleichzeitig f�r das Senden und Empfangen zust�ndig ist.\\ So kann die L�sung der aktiven Variante, wobei der Thread f�r die ben�tigte Zeit in \textit{sleep} gesetzt wird nicht verwendet werden.\\
			Aus diesem Grund wird nach jedem Senden an den Client der Zeitstempel abgespeichert. Bevor ein neues Paket gesendet wird, wird kontrolliert, ob die Differenz der aktuellen Zeit und dem letzten Sendezeitpunkt der gew�nschten Pause entspricht.\\
			Die Pausen werden �quivalent zu der aktiven Variante definiert.
			
	
\chapter{Evaluation der Passiven Variante}
	
	\section{Tests mit Testserver}
			
	
	\section{Schwierigkeiten bei der Verwendung einer realer Webseite}
	
			\subsection{HTTPS}
				Der Proxy ist nicht f�r die Verwendung von HTTPS ausgelegt, wodurch hier nur http Webseiten abgerufen werden k�nnen. 
	
			\subsection{Generierung von gen�gend Nachrichten}
				F�r den Covert Channel wird ein kontinuierlicher Datenstrom ben�tigt. Durch den verbindungsorientierten Aufbau von TCP ist es schwierig die Daten zu puffern und sp�ter auf ein mal abzusenden, da auf die ACK Nachricht gewartet wird.\\
				Um k�nstlich einen hohen Netzwerkverkehr zu generieren, kann beim Empfangen des Sockets eine geringe Datenmenge angegeben.\\
				Hier wird die maximale Paketgr��e auf 256 Byte eingegrenzt. Dadurch wird gerade beim Empfangen von Bildern ein enormer Netzwerkverkehr erzeugt, welcher f�r den Covert Channel genutzt werden kann.
				
	
			\subsection{Parallele Datenabfrage}	
				Wie vorhergehnd bereits beschrieben �ffnet einen normale Webserver gleich mehrerer Sockets um parallel Daten zu empfangen. Diese Nachrichtenpakete, der einzelnen Sockets k�nnen nicht vom Client auseinander gehalten werden, da dieser nur den Source-Port betrachtet. Dieser ist jedoch bei allen gleich.\\
				In Bild 7.1 ist ein Ausschnitt aus den Firefox Development Tool zu sehen. Hier wird die �bertragungsdauer der einzelnen Dateien gezeigt. Dabei f�llt auf, dass bei der hier verwendeten Webseite der \textit{PH-Weingarten}, 5 CSS Dateien gleichzeitig �bertragen werden.\\
				Diese 5 daraus entstehenden Covert Channel �berschneiden sich bei der Interpretation im Client und sind so nicht mehr auswertbar.\\
				Der Covert Channel funktioniert nur, wenn genau eine Datei �bertragen wird.
				
				\begin{figure}[h!]
					\centering
					\includegraphics*[height=7cm]{bilder/ParallelProblem.png}
					\caption{Parallele Datenabfrage}
					\label{fig43}
				\end{figure}
				
				Als m�gliche L�sung kann der Client zur Ber�cksichtigung der Sequenznummer erweitert werden um die verschiedenen Covert Channel zu unterscheiden.
	
	
		\section{Test mit realen Servern}


\chapter{Optimierung}
Buffer
client detect break self
Pausenverkleinerung/ tolleranzanpassung



\chapter{Bewertung der Ergebnisse}


paketmenge berechnen
fehler/informationsgehalt
durchschnit der Daten ausrechnen





						  % Ein weiteres Kapitel des Hauptteils
		  		  % leichter gefunden zu werden
\chapter{Schlussbemerkungen und Ausblick}


%%% Local Variables: 
%%% mode: latex
%%% TeX-master: "Bachelorarbeit"
%%% End: 
               % Schluss

%%% Anhang %%%          Nummerierung beginnend bei A
\appendix                       % Anhang
 \chapter{Ein Kapitel des Anhangs}
\label{cha:anhang}




%%% Local Variables: 
%%% mode: latex
%%% TeX-master: "Bachelorarbeit"
%%% End: 
               % Erstes Kapitel des Anhangs

%%% Verzeichnisse %%%
\begin{spacing}{1.0}            % Verzeichnisse werden mit einzeiligem Abstand gesetzt

% \listoffigures                   % Abbildungsverzeichnis (optional)
% \listoftables                    % Tabellenverzeichnis (optional)

%Glossar ausgeben
\printglossary[style=altlist,title=Glossar]

%Abk�rzungen ausgeben
\renewcommand{\acronymname}{Abk�rzungsverzeichnis}
\printglossary[type=\acronymtype,style=long]

%Symbole ausgeben
\printglossary[type=symbolslist,style=long]

\bibliographystyle{geralpha}       % Look&Feel vom Literaturverzeichnis
\bibliography{bib}                 % Literaturverzeichnis
\addcontentsline {toc}{chapter}{Stichwortverzeichnis} % Stichwortverzeichnis soll im Inhaltsverzeichnis auftauchen (optional)
\printindex % Stichwortverzeichnis endgueltig anzeigen

\end{spacing}

\end{document}

%%% WICHTIG:
%% Um den Glossareintrag (Abk�rzungsverzeichnis richtig darstellen zu k�nnen, muss makeindex mit dem Parameter "-s Bachelorarbeit.ist -t Bachelorarbeit.alg -o
%% Bachelorarbeit.acr Bachelorarbeit.acn" aufgerufen werden --> Einstellung im TeXnicCenter unter Ausgabe -> Ausgabeprofil definieren -> LaTeX=>PDF -> Nachbearbeitung
%% Unter Postprozessoren neuen Eintrag anlegen, z.B. Makeindex1. Unter Anwendung makeindex.exe ausw�hlen. (c:\programme\miktex2.7\miktex\bin\makeindex.exe)
%% unter Argumente die obige Parameterzeile eintragen. Bei anderer TeX-Distri makeindex suchen. Unter Linux ein shell-script erstellen, das makeindex mit 
%% den Parametern aufruft.
%% makeindex erneut starten mit folgender Parameterzeile:  " -s Bachelorarbeit.ist -t Bachelorarbeit.glg -o Bachelorarbeit.gls Bachelorarbeit.glo"
%% Analog zu oben im TeXnicCenter einen weiteren Eintrag Makeindex2 erstellen. In Linux einen weiteren makeindex-Aufruf im Script hinzuf�gen.
%% makeindex erneut starten mit folgender Parameterzeile:  " -s Bachelorarbeit.ist -t Bachelorarbeit.slg -o Bachelorarbeit.syi Bachelorarbeit.syg"
%% Analog zu oben im TeXnicCenter einen weiteren Eintrag Makeindex3 erstellen. In Linux einen weiteren makeindex-Aufruf im Script hinzuf�gen.

%% Bachelorarbeit muss durch den Namen der Hauptdatei ausgetauscht werden. Hauptdatei unter Windows mindestens 5 Mal compilieren, dann betrachten

%% Local Variables:
%% mode: latex
%% TeX-master:
%% End:
