%Kapitel des Hauptteils

\chapter {Grundlagen}  %Name des Kapitels
\label{cha:grundlagen} %Label des Kapitels



\section{Datenschutz}

Der Datenschutz ist ein �berbegriff f�r das in Gesetzte festgelegten Recht, dass jede Person �ber die Preisgabe der personenbezogenen Daten bestimmen kann. Die Bundesbeauftragten f�r den Datenschutz und die Informationssicherheit (BfDI) definieren den Datenschutz wie folgt:\\\\
\textit{,,Datenschutz garantiert jedem B�rger Schutz vor missbr�uchlicher Datenverarbeitung, das Recht auf informationelle Selbstbestimmung und den Schutz der Privatsph�re''}
 \cite{BfDI}\\\\
 Das bedeutet, dass jeder der personenbezogene Daten, ohne Zustimmung des Betreffenden speichert oder weiterverarbeitet vor Gericht angeklagt werden kann.
 Damit dies nicht passiert haben die meisten Institute die mit personenbezogenen Daten umgehen einen Datenschutzbeauftragten, der die Einhaltung dieser Gesetzte �berwacht.
 




\section{Datensicherheit}

Im Gegensatz zu dem Datenschutz bezieht sich die Datensicherheit nicht nur auf die personenbezogenen Daten, sondern auf alle Daten. Die Aufgabe der Datensicherheit werden durch das CIA-Prinzip beschreiben.\\ Zur Datensicherheit geh�ren nach diesem Prinzip alle Ma�nahmen, die die \textbf{C}onfidentiality, \textbf{I}ntegrity und \textbf{A}vailability (Vertraulichkeit, Integrit�t, Verf�gbarkeit) gew�hrleisten.\cite{CIA}
Eine zus�tzliche Aufgabe ist die Sicherstellung der Authentizit�t.
Viele dieser Aufgaben werden mit Hilfe der Kryptographie realisiert und umgesetzt. 


\section{Vertraulichkeit}
Die Vertraulichkeit ist dann gew�hrleistet, wenn die Daten nicht von unbefugten Personen eingesehen werden k�nnen. Es muss also ein System verwendet werden, bei dem sich befugte Benutzer legitimieren k�nnen und unbefugte beim Interpretieren gehindert werden. \\
In den meisten F�llen wird dies durch eine Verschl�sselung (symmetrisch oder asymmetrisch) umgesetzt. Alle legitimierten Benutzter erhalten den Schl�ssel. Die Personen ohne Schl�ssel k�nnen die Informationen nicht entschl�sseln - die Vertraulichkeit ist so garantiert.\\
Optimal w�re, wenn nicht legitime Benutzer, auch nicht an die verschl�sselten Daten kommen w�rde.

\section{Integrit�t} 
Die Integrit�t besch�ftigt sich damit, dass Daten nicht unbemerkt ver�ndert oder abgef�lscht werden. So soll zum Beispiel sichergestellt werden, dass eine Nachricht genau so beim Empf�nger ankommt, wie sie abgesendet wurde.\\
Hierzu k�nnen Hash-Funktionen verwendet werden, die beim Ver�ndern der Nachricht einen anderen Wert ergeben w�rden. Dabei m�sste entweder die Hash-Funktion geheim sein oder der Hash-Wert verschl�sselt werden.
 

\section{Verf�gbarkeit} 


\section{Authentizit�t} 
Die Authentizit�t best�tigt, dass Daten auch von einer angegeben Informationsquelle stammen. Es ist ein Identit�tsbeweis des Absenders gegen�ber dem Empf�ngers.


\section{Kryptographie}
Kerckhoffs? Prinzip


\section{Symmetrische Verschl�sselung}

\section{Asymmetrische Verschl�sselung}

\section{Steganographie}

\section{Internet Protokolle}






 
%%% Local Variables: 
%%% mode: latex
%%% TeX-master: "Bachelorarbeit"
%%% End: 
