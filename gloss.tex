%% Symbole:
%%\newglossaryentry{symb:Name}{name=Symbolname, description={Beschreibung}, sort=alphabetisches Wort f�r die Einreihung, type=symbolslist}
\newglossaryentry{symb:Pi}{
name=$\pi$,
description={Die Kreiszahl.},
sort=symbolpi, type=symbolslist
}
%%Abk�rzungen:
%%\newacronym{Referenz}{Abk�rzung}{Beschreibung}
\newacronym{BSP}{BSP}{Beispiel}


%%Eine Abk�rzung mit Glossareintrag:
%%\newacronym{Referenz}{Abk�rzung}{Beschreibung\protect\glsadd{glos:Referenz}}
\newacronym{AD}{AD}{Active Directory\protect\glsadd{glos:AD}}

%%Glossareintrag:
%%\newglossaryentry{glos:Referenz}{name=Name, description={Beschreibung}}
\newglossaryentry{glos:AD}{
name=Active Directory,
description={Active Directory ist in einem Windows Server 2000, Windows
Server 2003, oder Windows Server 2008-Netzwerk der Verzeichnisdienst, 
der die zentrale Organisation und Verwaltung aller Netzwerkressourcen erlaubt. Es
erm�glicht den Benutzern �ber eine einzige zentrale Anmeldung den
Zugriff auf alle Ressourcen und den Administratoren die zentral
organisierte Verwaltung, transparent von der Netzwerktopologie und
den eingesetzten Netzwerkprotokollen. Das daf�r ben�tigte
Betriebssystem ist entweder Windows Server 2000, 
Windows Server 2003, oder Windows Server 2008, welches auf dem zentralen
Dom�nencontroller installiert wird. Dieser h�lt alle Daten des
Active Directory vor, wie z.B. Benutzernamen und
Kennw�rter.}
}

\newglossaryentry{glos:Glossareintrag}{name=Glossareintrag, description={Erweiterte Informationen zum
einem Wort oder einer Abk�rzung, �hnlich einem Eintrag im Duden.}}




%%% Local Variables: 
%%% mode: latex
%%% TeX-master: "Bachelorarbeit"
%%% End: 