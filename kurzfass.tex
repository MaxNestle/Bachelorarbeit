\addchap{Kurzfassung}
\label{cha:kurzfassung} 

In dieser Arbeit wird eine Methode entwickelt, die als Alternative zu oft verwendeten Verschl�sselungsverfahren dienen soll. Hierzu werden Anwendungen aus dem Bereich der Steganographie, aber auch so genannte Covert Channels analysiert. Die Verfahren werden evaluiert und die beste Alternative als Proof of Concept implementiert.\\
Bei dem ausgew�hlten Kommunikationsverfahren handelt es sich um einen zeitabh�ngigen Covert Channel. Die geheimen Daten werden durch die Manipulation der Zeitabst�nde zwischen den Datenpaketen �bertragen. Daf�r gibt es zwei M�glichkeiten: Zum einen kann der sendende Server die Manipulation aktiv vornehmen, zum anderen bedient man sich einer passiven Variante, bei der ein vorgeschalteter Proxy die Beeinflussung der Zeitintervalle �bernimmt. Zus�tzlich wird ein Client erstellt, der den Covert Channel auswertet.\\
Die entstanden L�sungsvariante, dient als Proof of Concept und best�tigt die Funktionalit�t eines zeitabh�ngigen Covert Channels. Hiermit k�nnen Daten ohne inhaltliche Ver�nderung der Netzwerkpakete �bertragen werden.\\
Heutzutage konzentrieren sich Angreifer meist auf das Dechiffrieren von verschl�sselten Nachrichten. Methoden wie Covert Channels sind dagegen nicht im Fokus.

%%% Local Variables: 
%%% mode: latex
%%% TeX-master: "Bachelorarbeit"
%%% End: 


