

\chapter{Was ist ein Covert-Channel?}
\label{cha:Was ist ein Covert-Channel?} %Label des Kapitels

\section{Definition}
%TODO
Covert-Channels sind Netzwerk Kan�le, die nicht offen praktiziert, erkl�rt, engagiert, angesammelt oder gezeigt werden. \cite{goltz2003covert}

Hierbei kann ein Kanal eine beliebige Methode sein Informationen zischen zwei Ger�ten �ber ein Netzwerk auszutauschen.
Bei Covert-Channels wird sich die gr��tm�glich M�he gegeben, diesen Informationskanal vor Dritten unbemerkt zu halten.\\
Bei steganografischen Covert-Channels werden Methoden aus der Steganografie verwendet um diesen Kanal zu verbergen.

In dieser Arbeit handelt es sich bei Covert-Channel immer um Kan�le, die mit einem steganografischem Verfahren verschl�sselt werden.

\section{Was zeichnend einen guten Covert-Channel aus?}


Bei den weit verbreiteten mathematischen Verschl�sselungen muss sich meistens keine sorgen gemachen werden, ob der Datenaustausch von Dritten entdeckt werden kann, da hier die Daten ohne richtigen Key nutzlos sind.\\
Bei den Covert-Chanels ist dies problematischer, da ein entdeckter Kanal in der Regel direkt interpretiert werden kann. Ein Covert-Channel lebt, wie der Name auch schon verr�t, davon wie gut dieser versteckt ist.\\ 
Dabei besitzt die Steganografie einen gro�en Vorteil: Ist das Verfahren zum Verstecken der Daten nicht bekannt, so ist es nahezu unm�glich den Covert-Channel zu finden, da nicht klar ist wo und nach was gesucht werden muss (Security by Obscurity).\\
Ist hingegen klar, um welches Verfahren es sich handelt und besteht die Vermutung, dass eine Kommunikation �ber einen versteckten Kanal stattfindet so kommt man leicht an die Informationen.\\
Um einen Covert-Channel zu Bewerten m�ssen folgende Aspekte betrachtet werden:\\

Der erste ist die F�higkeit, wie einfach sich ein Kanal verstecken l�sst. Hier flie�t zum Einen die Allgemeine Unauff�lligkeit des Covert-Channels ein. Aber auch die Eigenschaften des bereits herrschenden Netzwerkverkehrs, in den der Channel eingebettet werden soll.\\
Der zweite Aspekt ist die allgemeine Unbekanntheit des Verfahrens, sodass nicht nach einem m�glichen versteckten Kanal gesucht werden kann. Hier kann einen Methode zur individuellen Gestaltung des Channels eine Verbesserung bringen.\\
Nat�rlich muss auch die Daten�bertragungsrate betrachtet. Diese ist meistens sehr gering aber hier gibt es auch gro�e Unterschiede zwischen den einzelnen Verfahren. \\
Die Daten sollten

\section{Wozu sind Covert-Channels nicht geeignet?}

Hohe Daten�bertragung

\chapter{M�gliche steganografische Covert-Channel}
\label{cha:Covert-Channel} %Label des Kapitels

\section{Zeitabh�ngiger Covert-Chanel}



\section{Benutzung verschiedener Protokolle}



\section{Benutzung von unbenutzten Bits in Protokoll-Headern}



\section{Verwendung der Paketgr��e}